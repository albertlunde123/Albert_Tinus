\documentclass[working, oneside]{inputs/tuftebook}
%%%%%%%%%%%%%%%%%%%%
%% SUPER PREAMBLE %%
%%%%%%%%%%%%%%%%%%%%

\documentclass[a4paper]{article}
\usepackage[utf8]{inputenc}
\usepackage[T1]{fontenc} % Fonts and stuff
\usepackage{amsmath, amsfonts, mathtools, amsthm, amssymb} % math

\usepackage{fancyhdr} % Header, Footer etc.
\usepackage{adforn}

\pagestyle{fancy}
\fancyhf{}
\fancyhead[R]{Albert Lunde, John Faxe Jensen}
\fancyfoot[C]{\adforn{17}\quad\thepage\quad\adforn{45}}

\renewcommand{\headrule}{%
	\hrulefill
	 {\quad\adforn{21}\adforn{10}\adforn{49}\quad}%
	\hrulefill}
\renewcommand{\footrulewidth}{0pt}

%% Margin Control %%

\def\changemargin#1#2{\list{}{\rightmargin#2\leftmargin#1}\item[]}
\let\endchangemargin=\endlist


%%%%%%%%%%%%%%%%%%%%%%%%%%%%%%%%%%%%%%%%%%%%%%%%%%%%%%%%%%

% figure support

\usepackage{import}
\usepackage{pdfpages}
\usepackage{transparent}
\usepackage{xcolor}

\newcommand{\incfig}[2][1]{%
    \def\svgwidth{#1\columnwidth}
    \import{figures/}{#2.pdf_tex}
}

\pdfsuppresswarningpagegroup=1

%%%%%%%%%%%%%%%%%%%%%%%%%%%%%%%%%%%%%%%%%%%%%%%%%%%%%%%%%%

\usepackage{tikzsymbols} % Symbols
\usepackage{mdframed} % Boxes around theorem environments
\usepackage{thmtools}


% Exercise environment with surrounding box

\declaretheoremstyle[
    name=Exercise,
    mdframed={
  skipabove=0pt,
  skipbelow=20pt,
  innerleftmargin=10pt,
  innerrightmargin=10pt,
  innerbottommargin=7pt}
]{thmsty}
\declaretheorem[style=thmsty,numbered=no]{exercise}

% Solution environment, with coffee cup

\newenvironment{solution}
 {\renewcommand\qedsymbol{\tikzsymbolsuse{Coffeecup}}\begin{proof}[Solution]}
 {\end{proof}}

% Subexercise enviroment
%  \newenvironment{subexercise}[1]
%  {
% 	\begin{changemargin}{1.0cm}{1.0cm}
% 	\textbf{(#1)}\em
% 	}{
	% \end{changemargin}
	% } 

 \newenvironment{subexercise}[1]
 {\noindent
	 \textbf{(#1)} \quad \adforn{10} \quad \em
 }{}

% Mathematical typesetting stuff.

 \newcommand{\dd}{\mathrm{\textbf{d}}}



\usepackage{pdfpages}

\usepackage{lipsum}
\usepackage{parskip}
\usepackage{titletoc}

\usepackage{cmbright}
\usepackage{bm}

\begin{document}
\let\cleardoublepage\clearpage
\section*{Theory}
In this section we will examine the necessary in understanding the michelson-morley interferometer. At its most basic level, we are interested in understanding what happens when light waves collide. In this experiment we will be assuming that the colliding are identical in all aspects expect phase. Their wavelengths and frequencies are identical. Let us assume that our light wave moves along the optical axis, it may then described as,
\[
	\bm{E_i} = E_0 \cos\left( \omega t - kx \right) 
.\] 
\begin{marginfigure}:q
    \centering
    \incfig{fig1}
    \caption{When the light is the incident light hits the beamsplitter, part of it is reflected and the remainder transmitted. Each lightbeam then travels a distance before hitting a mirror. The difference between these distances affects their relative phasedifference. We call it $\Delta s$.}
    \label{fig:fig1}
\end{marginfigure}
Where $\omega$ is the frequency, $k$ the wave number and $E_0$ the amplitude of the wave.When the wave is measured, it has been transmitted and reflected once. We therefore multply the wave amplitude by the coefficients of transmission and reflection, given by the Fresnel Relations.\cite{grif}
 \[
	 \left|\bm{E_i}\right| = \sqrt{RT} \cdot E_0 \cdot \cos\left( \omega t + \rho _i \right) 
.\] 
Where $\rho_i$ is the phase of our wave, at the point where our detector lies. This phase is clearly related to the path length in the following way,
\[
\rho _i  =  \frac{2\pi}{\lambda} S_i
.\]
For our two waves we obtain,
\begin{align*}
	\left|\bm{E_1}\right| = \sqrt{RT} \cdot E_0 \cdot \cos\left( \omega t + \rho _1 \right) \\
	\left|\bm{E_2}\right| = \sqrt{RT} \cdot E_0 \cdot \cos\left( \omega t + \rho _2 \right) 
\end{align*}
Where we have used the fact that transmission and reflection does not impact the frequency of light
If the optics are aligned correctly, we will the be able to measure the overlapping wave on our detector. This wave is given as the sum of $\bm{E_1}$ and $\bm{E_2}$. It's intensity is,
\[
I = c\epsilon_0 \left| \bm{E_1}+\bm{E_2} \right| ^2 = c\epsilon_0RT \left( \cos\left( \omega t +\rho_1 \right) +\cos\left( \omega t +\rho_2 \right)   \right)^2
.\] 
In practice, we are only able to measure the temporal averaging of this, as the frequency is a small quantity. The average of a periodic function with period $\tau$ is,
\[
\left<f \right> = \frac{1}{\tau} \int_{0}^{\tau}f\left( t \right) dt  
.\]
This gives us,
\begin{align*}
\left<I \right> &=  \frac{1}{2\pi} \int_{0}^{2\pi} \left[ \cos\left( \omega t + \rho_1 \right) + \cos\left( \omega t + rho_2 \right)   \right] ^2 d \left( \omega t \right)  \\
&=  1+ \cos\left( \rho _1 - \rho_2  \right)  
.\end{align*}
Now let $ \Delta \rho = \rho_1 - \rho_2$ and also assume that the coeffections of transmissions and reflection equal $0.5$. We then obtain,
 \[
I = \frac{1}{4}c\epsilon_0 E_0^2 \left( 1 + \cos \Delta \rho  \right) 
.\]
\subsection*{Piezoelectric}
We make use of af piezoelectric ring chip, which has the property that it expands when the potential over it increases. This allows us to slightly increase and decrease the path difference the path difference $\Delta s$ in a controlled fashion. The relationship between the potential applied over the piezoelectric and the expansion is nearly linear, as seen in \textbf{fig. 2}. As a result, the path difference should be proportional to the voltage
\[
V \propto \Delta s
\]
It is particularly smart to change the voltage linearly (in time). The path difference will then change linearly as well (approximately). A measurement of the intensity, should then change sinusoidally in time.
\end{document}
