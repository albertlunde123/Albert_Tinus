\documentclass[working]{inputs/tuftebook}
%%%%%%%%%%%%%%%%%%%%
%% SUPER PREAMBLE %%
%%%%%%%%%%%%%%%%%%%%

\documentclass[a4paper]{article}
\usepackage[utf8]{inputenc}
\usepackage[T1]{fontenc} % Fonts and stuff
\usepackage{amsmath, amsfonts, mathtools, amsthm, amssymb} % math

\usepackage{fancyhdr} % Header, Footer etc.
\usepackage{adforn}

\pagestyle{fancy}
\fancyhf{}
\fancyhead[R]{Albert Lunde, John Faxe Jensen}
\fancyfoot[C]{\adforn{17}\quad\thepage\quad\adforn{45}}

\renewcommand{\headrule}{%
	\hrulefill
	 {\quad\adforn{21}\adforn{10}\adforn{49}\quad}%
	\hrulefill}
\renewcommand{\footrulewidth}{0pt}

%% Margin Control %%

\def\changemargin#1#2{\list{}{\rightmargin#2\leftmargin#1}\item[]}
\let\endchangemargin=\endlist


%%%%%%%%%%%%%%%%%%%%%%%%%%%%%%%%%%%%%%%%%%%%%%%%%%%%%%%%%%

% figure support

\usepackage{import}
\usepackage{pdfpages}
\usepackage{transparent}
\usepackage{xcolor}

\newcommand{\incfig}[2][1]{%
    \def\svgwidth{#1\columnwidth}
    \import{figures/}{#2.pdf_tex}
}

\pdfsuppresswarningpagegroup=1

%%%%%%%%%%%%%%%%%%%%%%%%%%%%%%%%%%%%%%%%%%%%%%%%%%%%%%%%%%

\usepackage{tikzsymbols} % Symbols
\usepackage{mdframed} % Boxes around theorem environments
\usepackage{thmtools}


% Exercise environment with surrounding box

\declaretheoremstyle[
    name=Exercise,
    mdframed={
  skipabove=0pt,
  skipbelow=20pt,
  innerleftmargin=10pt,
  innerrightmargin=10pt,
  innerbottommargin=7pt}
]{thmsty}
\declaretheorem[style=thmsty,numbered=no]{exercise}

% Solution environment, with coffee cup

\newenvironment{solution}
 {\renewcommand\qedsymbol{\tikzsymbolsuse{Coffeecup}}\begin{proof}[Solution]}
 {\end{proof}}

% Subexercise enviroment
%  \newenvironment{subexercise}[1]
%  {
% 	\begin{changemargin}{1.0cm}{1.0cm}
% 	\textbf{(#1)}\em
% 	}{
	% \end{changemargin}
	% } 

 \newenvironment{subexercise}[1]
 {\noindent
	 \textbf{(#1)} \quad \adforn{10} \quad \em
 }{}

% Mathematical typesetting stuff.

 \newcommand{\dd}{\mathrm{\textbf{d}}}



\usepackage{pdfpages}
\usepackage{efbox}


\usepackage{lipsum}
\usepackage{parskip}
\usepackage{titletoc}

\usepackage{cmbright}
\usepackage{bm}

\makeatletter
\newcommand{\globalcolor}[1]{%
  \color{#1}\global\let\default@color\current@color
}
\makeatother

\AtBeginDocument{\globalcolor{white}}

\definecolor{nord}{HTML}{313847}

\pagecolor{nord}


\begin{document}

\chapter*{Teori}
\subsubsection*{Snell's Law}
When light passes through a dielectrica, the angle of reflection is equal to the angle of incidence, whereas the angle refraction is given by Snell's law;
\begin{marginfigure}
    \incfig{snell}
    \caption{Depiction of Snell's Law. As the light crosses the barrier between the two dielectrica, its angle changes. This angle is called the \textit{angle of refraction}.}
    \label{fig:snell}
\end{marginfigure}
\[
	\boxed{n_a \sin \theta_a = n_b \sin \theta_b}
\]
Where $n_a$ and $n_b$ are known as the indexes of refraction, which are properties of different materials. A materials index of refraction $n$ is related to the speed at which light moves through the material.
\[
n = \frac{c}{v}
\]
\subsubsection{Polarized light}
An electromagnetic wave consists of an oscillating electric field and an oscillating magnetic field. Since these fields are perpendicular and the propagation direction is perpendicular to both fields it is a transverse wave. Every electromagnetic wave has a certain orientation in space which can be determined by using the right hand rule. This can also be used to find the polarization of the wave, where the direction of the electric field determines the polarization. E.g if the orientation of the electric field is in the y-direction the wave is called linearly polarized in the y-direction.
\\
The visible light emitted from natural light source such as a light bulb or a laser is unpolarized. This means that the light consists of waves that are polarized in every possible direction. The light can be polarized in a specific direction by using a device called a polarizer. This device acts as a filter allowing only waves polarized in a specific angle through. Relevant for this experiment is when putting a polarizer rotated to 45 degrees in front of the laser, the electric and magnetic field of the light passing through can be split into two equally large components.

\begin{marginfigure}
    \incfig{polarized}
    \caption{The top figure depicts S-polarized light, while the bottom figure depicts P-polarized light.}
    \label{fig:polarized}
\end{marginfigure}
\begin{marginfigure}
    \incfig{fresnel1}
    \caption{The proportion of light that is transmitted and reflected is described by Fresnel's relations. For both $S$- and $P$-polarized light, a greater proportion is transmitted for most angles. Notice, that when the dielectrica has a circular, the light can pass through without changing direction} 
    \label{fig:fresnel1}
\end{marginfigure}
\subsubsection{Brewster's angle}
At a specific angle determined by the index of refraction of the two materials, the reflected light is polarized in the perpendicular direction of the plane of incidence. The relation is given in Brewster's law for the polarizing angle:
\begin{align*}
    tan(\theta_p) = \frac{n_b}{n_a}
\end{align*}
\begin{marginfigure}
	\includegraphics[width = 1.1\textwidth]{figures/fresnel2}
	\caption{This plot displays the intensities of the transmitted and reflected light. The transmitted light dominates for most angles. At tne brewster angle, all light is reflected. In this case, the index of refraction is 1.5 (glass).}
\end{marginfigure}
\subsubsection{The critical angle}
Another angle of interest is the critical angle, which can be derived from Snell's law of refraction by setting the refraction angle to 90 degrees:
\begin{align*}
    sin(\theta_{crit})=\frac{n_b}{n_a}
\end{align*}

At this angle no light is transmitted through the material. Instead the refracted light would move parallel with the demarcation line of the two materials. If the angle gets larger then total internal reflection occurs.
\subsubsection{Fresnel's Relations}
The angles of reflection and refraction are only part of the story. We are also interested in knowing how much of the incident light is reflected and transmitted. These intensities are given by the Fresnel relations, of which there are four. These relations describe the intensities of reflected and transmitted light for parallel and perpendicular polarized light. They have the following form;
\begin{align*}
	R_p &= \frac{\tan^2 \left( \theta_1 - \theta_2  \right) }{\tan^2 \left( \theta_1 + \theta_2  \right) }\\
	T_p &= \frac{sin(2\theta_1) sin(2\theta_2)}{sin^2(\theta_1+\theta_2)\cdot cos^2(\theta_1-\theta_2)}\\
	R_s &= \frac{\sin^2\left( \theta_1 - \theta_2 \right) }{\sin^2\left( \theta_1 + \theta_2 \right) } \\
	T_s &= \frac{sin(2\theta_1)sin(2\theta_2)}{sin^2(\theta_1 + \theta_2)}
.\end{align*}

If the energy is conserved then the sum of the intensity of all the different kinds of light should be equal to the intensity of the laser.
\begin{align*}
    T_p+T_s+R_p+R_s=I_{Laser}
\end{align*}
\begin{figure}[ht]
    \centering
    \incfig{fresnel3}
    \caption{fresnel3}
    \label{fig:fresnel3}
\end{figure}
\subsection{Experimental setup}
The setup of the experiment is shown in the following figure:[Made by projectAlbert]. Imagetext[Setup]
\\

\subsubsection*{Material List}
\begin{itemize}
    \item Red Laser
    \item 2x Polarizer
    \item Dielectric
    \item Picoscope
    \item Turntable 
    \item Lens
    \item Light intensity detector with slit
\end{itemize}
The light from the laser moves through a polarizer set at an angle of 45\textdegree. This implies that the intensity of s- and p-polarized is equal. From the light impacts the dielectric where it is reflected and refracted. The light detector is moved to catch either the reflected or refracted. In front of the detector is a lens, which focuses the light, and anothera polarizer, which selects either s- or p-polarized light. The slit in front of the detector, changes the detectors sensitivity. In practice we move the turntable, keeping the dielectric fixed, until the intensity measured by the detector reaches a maximum. We then record the intensity, angle of incidence and angle of transmission/refraction.

\subsection{Course of action}
In the experiment we want to examine different phenomena. These are:
\begin{itemize}
    \item Index of refraction.
    \item The relationship between the polarized light and the angle.
    \item The intensity of the total refracted light.
    \item The intensity of the total transmitted light.
    \item Conservation of the total intensity.
    
\end{itemize}
In order to test Fresnel's Relations we must know the index of refraction for the dielectric. Snell's law states that the index of refraction can be found from relation between the incidence and transmission angles. We therefore note transmission angle, for different angles of incidence. 
\\

When testing the conservation of the total intensity we can turn the dielectric and then measure the s- and p-polarized intensity for the transmitted and the reflected light. If the conservation is true, their sum should be constant. By noting the angles of incidence and transmission, we also use these measurements to test the Fresnel equation for e.g. the intensity of the p-polarized transmitted light. 
\subsection{Results}
\subsection{Error propagation}
Measurements that contains errors that should be considered include:
\begin{itemize}
    \item The angle 
    \item The background noise
    \item The width of the slit
\end{itemize}
 The most influential error in this experiment, is the one associated with the measurement of the angle. The protractor has a scale of 1 degreee, so the error must be $\pm 0.5$ \textdegree. Which when you propagate it. There is no reason to include the error of the sensor because when it was held steady it yielded an error of 0. The circular slit was used for all experiments and by using trigonometry we could determine that the error in the light actually hitting the center of the detector was negligible compared to the error in the protractor. The background noise was measured to be around 0.1 V for all angles and therefore also considered to be negligible compared to the intensity of the laser.
\subsection{Discussion}

We had some general issues while doing this experiment that ended up influencing the amount of data we were able to collect and how familiar we became with the general setup of the experiment. The first week of the exercise we both had Corona and the second week we had a lot of issues with different components of the experiment not working properly. So probably we could have produced some better data if we did not have these problems. 
\subsection{Conclusion}

\end{document}
