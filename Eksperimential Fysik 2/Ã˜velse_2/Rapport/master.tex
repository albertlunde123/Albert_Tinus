\documentclass[working, oneside]{inputs/tuftebook}
\usepackage[utf8]{inputenc}
\usepackage[T1]{fontenc}
\usepackage{textcomp}

\usepackage{url}

\usepackage[
    sorting=nyt,
    style=alphabetic
]{biblatex}
\addbibresource{bibliography.bib}

\usepackage{hyperref}
\hypersetup{
    colorlinks,
    linkcolor={black},
    citecolor={black},
    urlcolor={blue!80!black}
}
\usepackage[noabbrev]{cleveref}

% Adds Bibliography, ... to Table of Contents
\usepackage[nottoc]{tocbibind}

\usepackage{graphicx}
\usepackage{float}
\usepackage[usenames,dvipsnames,svgnames]{xcolor}

% \usepackage{cmbright}

\usepackage{amsmath, amsfonts, mathtools, amsthm, amssymb}
\usepackage{mathrsfs}
\usepackage{cancel}

\newcommand\N{\ensuremath{\mathbb{N}}}
\newcommand\R{\ensuremath{\mathbb{R}}}
\newcommand\Z{\ensuremath{\mathbb{Z}}}
\renewcommand\O{\ensuremath{\emptyset}}
\newcommand\Q{\ensuremath{\mathbb{Q}}}
\newcommand\C{\ensuremath{\mathbb{C}}}
\let\implies\Rightarrow
\let\impliedby\Leftarrow
\let\iff\Leftrightarrow
\let\epsilon\varepsilon

\usepackage{tikz}
\usepackage{tikz-cd}

% theorems
\usepackage{thmtools}
\usepackage{thm-restate}
\usepackage[framemethod=TikZ]{mdframed}
\mdfsetup{skipabove=1em,skipbelow=0em, innertopmargin=12pt, innerbottommargin=8pt}

\theoremstyle{definition}

\makeatletter

\declaretheoremstyle[headfont=\bfseries\sffamily, bodyfont=\normalfont, mdframed={ nobreak } ]{thmgreenbox}
\declaretheoremstyle[headfont=\bfseries\sffamily, bodyfont=\normalfont, mdframed={ nobreak } ]{thmredbox}
\declaretheoremstyle[headfont=\bfseries\sffamily, bodyfont=\normalfont]{thmbluebox}
\declaretheoremstyle[headfont=\bfseries\sffamily, bodyfont=\normalfont]{thmblueline}
\declaretheoremstyle[headfont=\bfseries\sffamily, bodyfont=\normalfont, numbered=no, mdframed={ rightline=false, topline=false, bottomline=false, }, qed=\qedsymbol ]{thmproofbox}
\declaretheoremstyle[headfont=\bfseries\sffamily, bodyfont=\normalfont, numbered=no, mdframed={ nobreak, rightline=false, topline=false, bottomline=false } ]{thmexplanationbox}

\declaretheoremstyle[headfont=\bfseries\sffamily, bodyfont=\normalfont, numbered=no, mdframed={ nobreak, rightline=false, topline=false, bottomline=false } ]{thmexplanationbox}


\declaretheorem[numberwithin=chapter, style=thmgreenbox, name=Definition]{definition}
\declaretheorem[sibling=definition, style=thmredbox, name=Corollary]{corollary}
\declaretheorem[sibling=definition, style=thmredbox, name=Proposition]{prop}
\declaretheorem[sibling=definition, style=thmredbox, name=Theorem]{theorem}
\declaretheorem[sibling=definition, style=thmredbox, name=Lemma]{lemma}
\declaretheorem[sibling=definition, style=thmbluebox,  name=Example]{eg}
\declaretheorem[sibling=definition, style=thmbluebox,  name=Nonexample]{noneg}
\declaretheorem[sibling=definition, style=thmblueline, name=Remark]{remark}




\declaretheorem[numbered=no, style=thmexplanationbox, name=Proof]{explanation}
\declaretheorem[numbered=no, style=thmproofbox, name=Proof]{replacementproof}
\declaretheorem[style=thmbluebox,  numbered=no, name=Exercise]{ex}
\declaretheorem[style=thmblueline, numbered=no, name=Note]{note}

% \renewenvironment{proof}[1][\proofname]{\begin{replacementproof}}{\end{replacementproof}}

% \AtEndEnvironment{eg}{\null\hfill$\diamond$}%

\newtheorem*{uovt}{UOVT}
\newtheorem*{notation}{Notation}
\newtheorem*{previouslyseen}{As previously seen}
\newtheorem*{problem}{Problem}
\newtheorem*{observe}{Observe}
\newtheorem*{property}{Property}
\newtheorem*{intuition}{Intuition}


\declaretheoremstyle[
    headfont=\bfseries\sffamily\color{RawSienna!70!black}, bodyfont=\normalfont,
    mdframed={
        linewidth=2pt,
        rightline=false, topline=false, bottomline=false,
        linecolor=RawSienna, backgroundcolor=RawSienna!5,
    }
]{todo}
\declaretheorem[numbered=no, style=todo, name=TODO]{TODO}


\usepackage{etoolbox}
\AtEndEnvironment{vb}{\null\hfill$\diamond$}%
\AtEndEnvironment{intermezzo}{\null\hfill$\diamond$}%

% http://tex.stackexchange.com/questions/22119/how-can-i-change-the-spacing-before-theorems-with-amsthm
% \def\thm@space@setup{%
%   \thm@preskip=\parskip \thm@postskip=0pt
% }

\usepackage{xifthen}

\makeatother

% figure support (https://castel.dev/post/lecture-notes-2)
\usepackage{import}
\usepackage{xifthen}
\pdfminorversion=7
\usepackage{pdfpages}
\usepackage{transparent}


\makeatletter
\newif\ifworking
\@ifclasswith{tuftebook}{working}{\workingtrue}{\workingfalse}
\makeatother

\newcommand{\incfig}[2][1]{%
    % \ifworking{\makebox[0pt][c]{\color{gray}{\scriptsize\textsf{#2}}}}\fi%
    \def\svgwidth{#1\textwidth}
    \import{./figures/}{#2.pdf_tex}
}

\newcommand{\fullwidthincfig}[2][0.90]{%
    % \ifworking{\makebox[0pt][l]{\color{gray}{\scriptsize\textsf{#2}}}}\fi%
    \def\svgwidth{#1\paperwidth}
    \import{./figures/}{#2.pdf_tex}
}



\newcommand{\minifig}[2]{%
    \def\svgwidth{#1}%
    \begingroup%
    \setbox0=\hbox{\import{./figures/}{#2.pdf_tex}}%
    \parbox{\wd0}{\box0}\endgroup%
    \hspace*{0.2cm}
}

% %http://tex.stackexchange.com/questions/76273/multiple-pdfs-with-page-group-included-in-a-single-page-warning
\pdfsuppresswarningpagegroup=1

\newcommand\todo[1]{\ifworking {{\color{red}{#1}}} \else {}\fi}
\newcommand\charlotte[1]{\ifworking {{\color{blue}{#1}}} \else {}\fi}

\author{Gilles Castel}



\usepackage{multirow}
\def\block(#1,#2)#3{\multicolumn{#2}{c}{\multirow{#1}{*}{$ #3 $}}}

% \overfullrule=1mm

\newenvironment{myproof}[1][\proofname]{%
  \proof[\rm \bf #1]%
}{\endproof}

\usepackage{pdfpages}

\usepackage{lipsum}
\usepackage{parskip}
\usepackage{titletoc}

\usepackage{cmbright}
\usepackage{bm}

\begin{document}
\let\cleardoublepage\clearpage
\section*{Theory}
In this section we will examine the necessary in understanding the michelson-morley interferometer. At its most basic level, we are interested in understanding what happens when light waves collide. In this experiment we will be assuming that the colliding are identical in all aspects expect phase. Their wavelengths and frequencies are identical. Let us assume that our light wave moves along the optical axis, it may then described as,
\[
	\bm{E_i} = E_0 \cos\left( \omega t - kx \right) 
.\] 
\begin{marginfigure}
    \centering
    \incfig{fig1}
    \caption{When the light is the incident light hits the beamsplitter, part of it is reflected and the remainder transmitted. Each lightbeam then travels a distance before hitting a mirror. The difference between these distances affects their relative phasedifference. We call it $\Delta s$.}
    \label{fig:fig1}
\end{marginfigure}
Where $\omega$ is the frequency, $k$ the wave number and $E_0$ the amplitude of the wave.When the wave is measured, it has been transmitted and reflected once. We therefore multply the wave amplitude by the coefficients of transmission and reflection, given by the Fresnel Relations.\cite{grif}
 \[
	 \left|\bm{E_i}\right| = \sqrt{RT} \cdot E_0 \cdot \cos\left( \omega t + \rho _i \right) 
.\] 
Where $\rho_i$ is the phase of our wave, at the point where our detector lies. This phase is clearly related to the path length in the following way,
\[
\rho _i  =  \frac{2\pi}{\lambda} S_i
.\]
For our two waves we obtain,
\begin{align*}
	\left|\bm{E_1}\right| = \sqrt{RT} \cdot E_0 \cdot \cos\left( \omega t + \rho _1 \right) \\
	\left|\bm{E_2}\right| = \sqrt{RT} \cdot E_0 \cdot \cos\left( \omega t + \rho _2 \right) 
\end{align*}
Where we have used the fact that transmission and reflection does not impact the frequency of light.
If the optics are aligned correctly, we will the be able to measure the overlapping wave on our detector. This wave is given as the sum of $\bm{E_1}$ and $\bm{E_2}$. It's intensity is,
\[
I = c\epsilon_0 \left| \bm{E_1}+\bm{E_2} \right| ^2 = c\epsilon_0RT \left( \cos\left( \omega t +\rho_1 \right) +\cos\left( \omega t +\rho_2 \right)   \right)^2
.\] 
In practice, we are only able to measure the temporal averaging of this, as the frequency is a small quantity. The average of a periodic function with period $\tau$ is,
\[
\left<f \right> = \frac{1}{\tau} \int_{0}^{\tau}f\left( t \right) dt  
.\]
This gives us,
\begin{align*}
\left<I \right> &=  \frac{1}{2\pi} \int_{0}^{2\pi} \left[ \cos\left( \omega t + \rho_1 \right) + \cos\left( \omega t + rho_2 \right)   \right] ^2 d \left( \omega t \right)  \\
&=  1+ \cos\left( \rho _1 - \rho_2  \right)  
.\end{align*}
Now let $ \Delta \rho = \rho_1 - \rho_2$ and also assume that the coeffections of transmissions and reflection equal $0.5$. We then obtain,
 \[
I = \frac{1}{4}c\epsilon_0 E_0^2 \left( 1 + \cos \Delta \rho  \right) 
.\] 
\end{document}
