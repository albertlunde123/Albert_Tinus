%%%%%%%%%%%%%%%%%%%%
%% SUPER PREAMBLE %%
%%%%%%%%%%%%%%%%%%%%

\documentclass[a4paper]{article}
\usepackage[utf8]{inputenc}
\usepackage[T1]{fontenc} % Fonts and stuff
\usepackage{amsmath, amsfonts, mathtools, amsthm, amssymb} % math

\usepackage{fancyhdr} % Header, Footer etc.
\usepackage{adforn}

\pagestyle{fancy}
\fancyhf{}
\fancyhead[R]{Albert Lunde, John Faxe Jensen}
\fancyfoot[C]{\adforn{17}\quad\thepage\quad\adforn{45}}

\renewcommand{\headrule}{%
	\hrulefill
	 {\quad\adforn{21}\adforn{10}\adforn{49}\quad}%
	\hrulefill}
\renewcommand{\footrulewidth}{0pt}

%% Margin Control %%

\def\changemargin#1#2{\list{}{\rightmargin#2\leftmargin#1}\item[]}
\let\endchangemargin=\endlist


%%%%%%%%%%%%%%%%%%%%%%%%%%%%%%%%%%%%%%%%%%%%%%%%%%%%%%%%%%

% figure support

\usepackage{import}
\usepackage{pdfpages}
\usepackage{transparent}
\usepackage{xcolor}

\newcommand{\incfig}[2][1]{%
    \def\svgwidth{#1\columnwidth}
    \import{figures/}{#2.pdf_tex}
}

\pdfsuppresswarningpagegroup=1

%%%%%%%%%%%%%%%%%%%%%%%%%%%%%%%%%%%%%%%%%%%%%%%%%%%%%%%%%%

\usepackage{tikzsymbols} % Symbols
\usepackage{mdframed} % Boxes around theorem environments
\usepackage{thmtools}


% Exercise environment with surrounding box

\declaretheoremstyle[
    name=Exercise,
    mdframed={
  skipabove=0pt,
  skipbelow=20pt,
  innerleftmargin=10pt,
  innerrightmargin=10pt,
  innerbottommargin=7pt}
]{thmsty}
\declaretheorem[style=thmsty,numbered=no]{exercise}

% Solution environment, with coffee cup

\newenvironment{solution}
 {\renewcommand\qedsymbol{\tikzsymbolsuse{Coffeecup}}\begin{proof}[Solution]}
 {\end{proof}}

% Subexercise enviroment
%  \newenvironment{subexercise}[1]
%  {
% 	\begin{changemargin}{1.0cm}{1.0cm}
% 	\textbf{(#1)}\em
% 	}{
	% \end{changemargin}
	% } 

 \newenvironment{subexercise}[1]
 {\noindent
	 \textbf{(#1)} \quad \adforn{10} \quad \em
 }{}

% Mathematical typesetting stuff.

 \newcommand{\dd}{\mathrm{\textbf{d}}}



\title{\vspace{-1cm}Pendul\vspace{-1cm}}
\author{}
\date{}

\begin{document}
\maketitle
\thispagestyle{fancy}
\begin{figure}[ht]
    \centering
    \incfig{pendul}
    \caption{pendul}
    \label{fig:pendul}
\end{figure}
\begin{figure}[ht]
    \centering
    \incfig{pendul2}
    \caption{pendul2}
    \label{fig:pendul2}
\end{figure}
Hvis vi antager at friktionskraften er $0$. Vil der kun være én kraft på pendulet,
 \[
F = m\cdot g\cdot \sin \theta
.\]
Kraftmomentet på pendulet vil være afstanden fra pivot-punktet til center of mass ganget med den vinkelrette komponent af kraften. 
\[
	\tau = -(mg)\sin \theta \cdot R_{cm}
.\] 
Hvis jeg nu også laver en lille-vinkel antagelse,
\[
	\tau = -(mg) \cdot \theta \cdot R_{cm}
.\]
Jeg ved desuden at kraftmomentet er,
\[
	I \frac{\dd^2 \theta}{\dd t^2} = -(mg)\cdot \theta(t)\cdot  R_{cm}
.\] 
 Vi vil gerne finde en funktion $\theta(t)$, som opfylder denne ligning. Jeg betrager funktionen $\theta (t) = \frac{1}{I}\cdot \sin(t\cdot \sqrt{mgR_{cm}}+\lambda )+c$.
\begin{align*}
	\frac{\dd \theta^2}{\dd t^2} \theta (t) &= \frac{\dd \theta}{\dd t}\frac{1}{I}\cdot \sqrt{mgR_{cm}}\left( -\cos(t\cdot \sqrt{mgR_{cm}} +\lambda \right)  \\
						&= \frac{-mgR_{cm}}{I}\sin(t\cdot \sqrt{mgR_{cm}} +\lambda) \\
.\end{align*}
Og ser at denne funktion opfylder kriteriet. I en analyse $\sin(t)$ ser jeg at $\sin(t) = t$ approksimationen holder nogenlunde for vinkler mindre end $0.40$ radianer.
\begin{figure}[ht]
    \centering
    \incfig{massemidtpunkts_bestemmelse}
    \caption{Massemidtpunkt}
    \label{fig:massemidtpunkt}
\end{figure}
\newpage
\begin{subexercise}{Bestemmelse af massemidtpunktet}
Her følger de teoretiske overvejelse, til bestemmelse af pendulets massemidtpunkt.
\end{subexercise}\newline
Se figur 3. Der er 2 krafter på pendulet, en tension fra snoren og tyngdekraften. Tyngdekraften virker i pendulets massemidtpunkt, mens tension virker der hvor snoren sidder fast. Disse giver begge anledning til et kraftmoment, men da pendulet er i hvile vil det samlede kraftmoment være 0.
\[
\tau = F_wR_w + F_TR_T
.\] 
Nu definerer jeg $R_T = R_0 + \Delta R$
\begin{align*}
	\frac{1}{F_T}&= \frac{-\left( R_0+\Delta R \right) }{R_wF_w} \\
	\frac{1}{F_T}&= \frac{-R_0}{R_wF_w}-\frac{\Delta R}{R_wF_w}
.\end{align*} 
\end{document}
