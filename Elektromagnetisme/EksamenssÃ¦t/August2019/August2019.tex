\usepackage[utf8]{inputenc}
\usepackage[T1]{fontenc}
\usepackage{textcomp}

\usepackage{url}

\usepackage[
    sorting=nyt,
    style=alphabetic
]{biblatex}
\addbibresource{bibliography.bib}

\usepackage{hyperref}
\hypersetup{
    colorlinks,
    linkcolor={black},
    citecolor={black},
    urlcolor={blue!80!black}
}
\usepackage[noabbrev]{cleveref}

% Adds Bibliography, ... to Table of Contents
\usepackage[nottoc]{tocbibind}

\usepackage{graphicx}
\usepackage{float}
\usepackage[usenames,dvipsnames,svgnames]{xcolor}

% \usepackage{cmbright}

\usepackage{amsmath, amsfonts, mathtools, amsthm, amssymb}
\usepackage{mathrsfs}
\usepackage{cancel}

\newcommand\N{\ensuremath{\mathbb{N}}}
\newcommand\R{\ensuremath{\mathbb{R}}}
\newcommand\Z{\ensuremath{\mathbb{Z}}}
\renewcommand\O{\ensuremath{\emptyset}}
\newcommand\Q{\ensuremath{\mathbb{Q}}}
\newcommand\C{\ensuremath{\mathbb{C}}}
\let\implies\Rightarrow
\let\impliedby\Leftarrow
\let\iff\Leftrightarrow
\let\epsilon\varepsilon

\usepackage{tikz}
\usepackage{tikz-cd}

% theorems
\usepackage{thmtools}
\usepackage{thm-restate}
\usepackage[framemethod=TikZ]{mdframed}
\mdfsetup{skipabove=1em,skipbelow=0em, innertopmargin=12pt, innerbottommargin=8pt}

\theoremstyle{definition}

\makeatletter

\declaretheoremstyle[headfont=\bfseries\sffamily, bodyfont=\normalfont, mdframed={ nobreak } ]{thmgreenbox}
\declaretheoremstyle[headfont=\bfseries\sffamily, bodyfont=\normalfont, mdframed={ nobreak } ]{thmredbox}
\declaretheoremstyle[headfont=\bfseries\sffamily, bodyfont=\normalfont]{thmbluebox}
\declaretheoremstyle[headfont=\bfseries\sffamily, bodyfont=\normalfont]{thmblueline}
\declaretheoremstyle[headfont=\bfseries\sffamily, bodyfont=\normalfont, numbered=no, mdframed={ rightline=false, topline=false, bottomline=false, }, qed=\qedsymbol ]{thmproofbox}
\declaretheoremstyle[headfont=\bfseries\sffamily, bodyfont=\normalfont, numbered=no, mdframed={ nobreak, rightline=false, topline=false, bottomline=false } ]{thmexplanationbox}

\declaretheoremstyle[headfont=\bfseries\sffamily, bodyfont=\normalfont, numbered=no, mdframed={ nobreak, rightline=false, topline=false, bottomline=false } ]{thmexplanationbox}


\declaretheorem[numberwithin=chapter, style=thmgreenbox, name=Definition]{definition}
\declaretheorem[sibling=definition, style=thmredbox, name=Corollary]{corollary}
\declaretheorem[sibling=definition, style=thmredbox, name=Proposition]{prop}
\declaretheorem[sibling=definition, style=thmredbox, name=Theorem]{theorem}
\declaretheorem[sibling=definition, style=thmredbox, name=Lemma]{lemma}
\declaretheorem[sibling=definition, style=thmbluebox,  name=Example]{eg}
\declaretheorem[sibling=definition, style=thmbluebox,  name=Nonexample]{noneg}
\declaretheorem[sibling=definition, style=thmblueline, name=Remark]{remark}




\declaretheorem[numbered=no, style=thmexplanationbox, name=Proof]{explanation}
\declaretheorem[numbered=no, style=thmproofbox, name=Proof]{replacementproof}
\declaretheorem[style=thmbluebox,  numbered=no, name=Exercise]{ex}
\declaretheorem[style=thmblueline, numbered=no, name=Note]{note}

% \renewenvironment{proof}[1][\proofname]{\begin{replacementproof}}{\end{replacementproof}}

% \AtEndEnvironment{eg}{\null\hfill$\diamond$}%

\newtheorem*{uovt}{UOVT}
\newtheorem*{notation}{Notation}
\newtheorem*{previouslyseen}{As previously seen}
\newtheorem*{problem}{Problem}
\newtheorem*{observe}{Observe}
\newtheorem*{property}{Property}
\newtheorem*{intuition}{Intuition}


\declaretheoremstyle[
    headfont=\bfseries\sffamily\color{RawSienna!70!black}, bodyfont=\normalfont,
    mdframed={
        linewidth=2pt,
        rightline=false, topline=false, bottomline=false,
        linecolor=RawSienna, backgroundcolor=RawSienna!5,
    }
]{todo}
\declaretheorem[numbered=no, style=todo, name=TODO]{TODO}


\usepackage{etoolbox}
\AtEndEnvironment{vb}{\null\hfill$\diamond$}%
\AtEndEnvironment{intermezzo}{\null\hfill$\diamond$}%

% http://tex.stackexchange.com/questions/22119/how-can-i-change-the-spacing-before-theorems-with-amsthm
% \def\thm@space@setup{%
%   \thm@preskip=\parskip \thm@postskip=0pt
% }

\usepackage{xifthen}

\makeatother

% figure support (https://castel.dev/post/lecture-notes-2)
\usepackage{import}
\usepackage{xifthen}
\pdfminorversion=7
\usepackage{pdfpages}
\usepackage{transparent}


\makeatletter
\newif\ifworking
\@ifclasswith{tuftebook}{working}{\workingtrue}{\workingfalse}
\makeatother

\newcommand{\incfig}[2][1]{%
    % \ifworking{\makebox[0pt][c]{\color{gray}{\scriptsize\textsf{#2}}}}\fi%
    \def\svgwidth{#1\textwidth}
    \import{./figures/}{#2.pdf_tex}
}

\newcommand{\fullwidthincfig}[2][0.90]{%
    % \ifworking{\makebox[0pt][l]{\color{gray}{\scriptsize\textsf{#2}}}}\fi%
    \def\svgwidth{#1\paperwidth}
    \import{./figures/}{#2.pdf_tex}
}



\newcommand{\minifig}[2]{%
    \def\svgwidth{#1}%
    \begingroup%
    \setbox0=\hbox{\import{./figures/}{#2.pdf_tex}}%
    \parbox{\wd0}{\box0}\endgroup%
    \hspace*{0.2cm}
}

% %http://tex.stackexchange.com/questions/76273/multiple-pdfs-with-page-group-included-in-a-single-page-warning
\pdfsuppresswarningpagegroup=1

\newcommand\todo[1]{\ifworking {{\color{red}{#1}}} \else {}\fi}
\newcommand\charlotte[1]{\ifworking {{\color{blue}{#1}}} \else {}\fi}

\author{Gilles Castel}



\usepackage{multirow}
\def\block(#1,#2)#3{\multicolumn{#2}{c}{\multirow{#1}{*}{$ #3 $}}}

% \overfullrule=1mm

\newenvironment{myproof}[1][\proofname]{%
  \proof[\rm \bf #1]%
}{\endproof}

\title{\vspace{-1cm}\efbox[margin = 15pt]{August - 2019}\vspace{-1cm}}
\author{}
\date{}

\begin{document}
\pagecolor{color1}
\maketitle
\thispagestyle{fancy}
\begin{exercise}[Opgave 1]
Vi betragter det elektrostatiske arrangement vist på Figur 1. En metalkugle
med radius $a$ og centereret i origo bærer en ladning $Q (Q > 0)$. En metalskal,
koncentrisk med kuglen og med indre radius b og ydre radius c, bærer en
samlet ladning $−2Q$. Der er vakuum i områdene $a < r < b$ og $r > c$, hvor $r$
er afstanden til origo, og den dielektriske permittivitet er lig  0 overalt.
\end{exercise}
\begin{figure}[ht]
    \centering
    \incfig[0.4]{opgave1}
    \label{fig:opgave1}
\end{figure}
\begin{subexercise}[a]
(a) Angiv overfladeladningstætheden på overfladene $r = a$, $r = b$ og $r = c$.
Bestem retningen og størrelsen af det elektriske felt i områderne $r < a$,
$a < r < b$, $b < r < c$ og $r > c$.
\end{subexercise}
\begin{solution}
	Alt ladning sidder på overfladen,
	\[
	\sigma_a = \frac{Q}{4\pi a^2}
	.\] 
	Ladningen på indersiden af den yderste sfære skal modsvarer ladningen i den inderste sfære. Da $Q_enc$ skal være 0.
	 \[
	\sigma_b = \frac{-Q}{4\pi b^2}
	.\] 
	Den resterende ladning sidder i $c$.
	 \[
	\sigma_c = \frac{-Q}{4\pi c^2}
	.\]
	Det elektriske felt er radiært, eller med andre ord,
	\[
	\vec{\mathbf{E}}\left( r \right) = E\left( r \right) \hat{r} 
	.\] 
	$E$-feltet inde i en leder er altid $0$, mens E-feltet udenfor en sfære eller sfærisk skal er identisk med $E$-feltet fra en punktpartikel.
	\[
	\vec{\mathbf{E}} \left( r \right) = 
	\begin{cases}
		0 \quad & r<a \\
		\frac{1}{4\pi \epsilon_0}\frac{Q}{r^2} \quad & a<r<b \\
		0 \quad & b<r<c \\
		\frac{1}{4\pi \epsilon_0}\frac{-Q}{r^2} \quad & r>c \\
	\end{cases}
	.\] 
\end{solution}
\begin{exercise}[Opgave 2]
En tynd ladet skive med radius $R$ bærer på den ene side en overfladeslad-
ningstæthed $\sigma = \alpha r$, hvor $\alpha $ er en positiv konstant og $r$ afstanden til skivens
centrum $O$ (Figur 2). Den dielektriske permittivitet er lig $\epsilon_0$ overalt.
\end{exercise}
\begin{figure}[ht]
    \centering
    \incfig[0.6]{opgave2}
    \label{fig:opgave2}
\end{figure}
\begin{subexercise}[a]
	Bestem den totale ladning båret af skiven, såvel som det elektriske potential i skivens centrum (udtrykt ved $\alpha $, $R$ og $\epsilon_0$ ).
\end{subexercise}
\begin{solution}
Jeg betragter $dQ$-elementer i en afstand $r$ og integrerer fra $0 \to R$. Ladningen i ét element,
\[
dQ = 2\pi r \rho\left( r \right) \dd r
.\]
Nu integrerer jeg,
\[
Q = 2\pi \alpha \int_{0}^{R} r^2 \, \dd r= \frac{2\pi \alpha }{3} r^3
.\] 
Til at bestemme det elektriske potentiale, bruger jeg følgende formel fra bogen, som stammer fra eksempel 23.11,
\[
V = \frac{1}{4\pi\epsilon_0}\frac{Q}{\sqrt{x^2+a^2} }
.\]
Som er det elektriske i en afstand $x$ fra en ring med ladning $Q$. Jeg betragter nu ringe med radius $r$ og sætter $x=0$, da jeg er interesseret i at finde potentialet i origo. Mine $\dd V$ elementer bliver dermed,
\[
\dd V = \frac{1}{4\pi\epsilon_0} \frac{\dd Q}{r} = \frac{1}{4\pi\epsilon_0} \frac{2\pi \alpha r^2 \dd r}{r}
.\] 
Jeg integrerer dette fra $0\to R$.
\[
V = \frac{1}{2} \frac{\alpha }{\epsilon_0}\int_{0}^{R} r\, \dd r = \frac{\alpha }{4\epsilon_0}R^2
.\] 
\end{solution}
\begin{exercise}[Opgave 3]
Vi betragter det elektriske kredsløb vist på Figur 3. Kredsløbet består af en
variabel emk kilde, der leverer en periodisk spænding $v(t) = V \cos(\omega t)$, hvor
V er spændingens maksimale værdi og $\omega$ vinkelfrekvensen. Kilden er koblet
til to modstande med modstand $2R$, to kapacitorer med kapacitans $C /2$ og
en ideel induktor med induktans$ L$. Vi ser bort fra fælles induktans effekter.
\end{exercise}
\begin{figure}[ht]
    \centering
    \incfig{opgave3}
    \caption{Opgave3}
    \label{fig:opgave3}
\end{figure}
\begin{subexercise}[a]
Forsimpl kredsløbet, angiv dens impedans og bestem $V_C /V$ , hvor $V_C$ er
den maksimale værdi af spændingen over kapacitorerne, $v_C (t)$. Hvilken
slags frekvensfilter fås ved at måle spændingen over kapacitorerne?
\end{subexercise}
\begin{solution}
Det forsimplede kredsløb, indeholder én kapacitor og én resistor. De ækvivalente størrelser er,
\begin{align*}
	C_{eq} &= \frac{C}{2} + \frac{C}{2} = C \\
	R_{eq} &= \left( \frac{1}{2R} + \frac{1}{2R} \right) ^{-1} = R 
.\end{align*}
Impedansen, findes ved af hjælp af formel (31.23),
\[
Z = \sqrt{R^2+\left[ \omega L - \left( 1 / \omega C \right)  \right]^2 }  
.\] 
\end{solution}
Jeg bruger nu følgende formler til at bestemme $V_C / V$,
\begin{align*}
V_C &=  \frac{I}{\omega C} \\
I &= \frac{V}{Z}
.\end{align*}
og får at,
\[
\frac{V_C}{V} = \frac{1}{\omega C Z}
.\] 
\begin{exercise}[Opgave 4]
Vi betragter det magnetostatiske arrangement vist på Figur 4a. En uendelig
lang og tynd ledning, hvori en konstant strøm $I$ løber i den positive z-retning,
ligger langs z-aksen. En rektangulær sløjfe med sidelændger $w$ og $h$, hvori en
konstant strøm $I_0$ løber i retningen mod uret, ligger stationært i (yz)-planen.
Den magnetiske permeabilitet er lig $\mu_0$ overalt.
\end{exercise}
\begin{figure}[ht]
    \centering
    \incfig{opgave4}
    \caption{Opgave4}
    \label{fig:opgave4}
\end{figure}
\begin{subexercise}[a]
Bestem retningen og størrelsen af den magnetiske kraft udøvet på sløjfen
af ledningen.
\end{subexercise}
\begin{solution}
Den magnetiske kraft på en leder er givet ved,
\[
\vec{\mathbf{F}} = \vec{\mathbf{I}} L \times \vec{\mathbf{B}} 
.\]
En uendelig ledning med strøm $I$ inducerer et magnetfelt af størrelsen,
 \[
B = \frac{\mu_0 I}{2\pi r}
.\]
I dette tilfælde peger det inducerede magnetfelt den positive $x$-retning. Jeg bemærker at pr. symmetri er kraften på de 2 horisontale sider i kredsløb identiske med modsat fortegn. Disse går dermed ud. Kraften på de 2 vertikale sider peger hver deres vej. Den inderste bliver påvirket af en kraft i $y$-retningen mens den yderste er i $-y$-retningen. Den resulterende kraft bliver dermed,
 \[
	 \vec{\mathbf{F}} = I'L\left( \frac{\mu_0 I}{2\pi} \right) \left( \frac{1}{d}-\frac{1}{d+w} \right) 
.\] 
\end{solution}
\begin{subexercise}[b]
Vi betragter nu et lignende arrangement, men hvor der kun løber en konstant
strøm $I$ i ledningen og hvor sløjfen roterer med konstant vinkelhastighed ω
omkring dens symmetriakse parallel med z-aksen (stiplet linje i Figur 4a),
således at vinklen mellem sløjfens plan og y-aksen til tiden $t$ er $\omega t$ (se
Figur 4b). Vi antager, at sløjfens dimensioner $h$ og $w$ er meget små i forhold
til afstanden til ledningen $d$ og kan betragte det magnetiske felt fra ledningen
til at være homogent ved sløjfen. Vi ser bort fra selvinduktans effekter og
sløjfens modstand er $R$. \\
Bestem retningen og størrelsen af den inducerede strøm i sløjfen til tiden $t$ hvor $t\in \left[ 0, \frac{\pi}{2\omega} \right] $
\end{subexercise}
\begin{solution}
Vi er interesseret i at bestemme den magnetiske flux til tiden $t$, da der netop gælder at,
 \[
\mathcal{E} = -\frac{\dd \Phi_B}{\dd t}
.\]
Den magnetiske flux er,
\[
\Phi_B = \int B\cdot dA
.\]
Da kredsløbet rotererer er den ortogonale del af arealet til tiden $t$,
\[
A\left( t \right) = \cos\left( \omega t \right) wh
.\]
Den magnetiske flux er dermed,
\[
\Phi_B = \frac{\mu_0 I}{2\pi d} \cos\left( \omega t \right) wh
.\] 
Den tidsafledte bliver,
\[
	\frac{\dd \Phi_B}{\dd t} = -\frac{mu_0 I\omega w h}{2\pi d}\sin \left( \omega t \right)
.\] 
Strømmen bliver dermed,
\[
I= \frac{mu_0 I\omega w h}{2\pi d}\frac{\sin \left( \omega t \right) }{R}
.\] 
\end{solution}
\begin{exercise}[Opgave 6]
Lys med bølgelængden $\lambda = 500 \text{ nm}$ sendes ind mod en skærm med to spalter
placeret i en indbyrdes afstand $d = 500 \text{ $\mu $m}$ (Fig. 6). En afbildningsskærm
er anbragt bag den første i afstanden $R$ = 1 m.
\end{exercise}
\begin{figure}[ht]
    \centering
    \incfig[0.4]{opgave6}
    \label{fig:opgave6}
\end{figure}
\begin{subexercise}[a]
	Løs følgende,
	\begin{itemize}
		\item Opskriv et udtryk for den resulterende intensitet på afbildningsskær-
men som funktion af positionen $y$ på skærmen, idet det oplyses at
intensiteten ved $y_0$, er 1 mW. Der kan ses bort fra diffraktions
effekter pga. den endelige udstrækning af de to spalter.
		\item  Skitser intensiteten som funktion af $y$ og opgiv positionen af det 5.maximum, $y_{1,5max}$ , og det 6. minimum, $y_{1,6min}$ , fra centerlinjen ved $y=0$.
		\item Bestem den bølgelængde $\lambda_2$ for hvilken det 5. interferens maximum, $y_{2,5max}$ er sammenfaldende med $y _{1,6min}$
	\end{itemize}

\end{subexercise}
\begin{solution}
Funktionen bestemmes med formel,
\[
I = I_0 \cos ^2 \frac{\phi}{2}
.\]
hvor,
\[
\phi = \frac{2\pi d}{\lambda}\sin \theta 
.\]
Da vi har med små vinkler at gøre er,
\[
\sin \theta = \frac{y}{R} 
.\]
Dette sættes ind og jeg får,
\[
I\left( y \right)  = I_0 \cos ^2 \left( \frac{\pi d y}{\lambda R} \right) 
.\]
Med et script bestemmes alle minima og maksima i intensiteten med formelen,
\[
y_m = \frac{Rm \lambda}{d}
.\]
I den sidste opgave sættes,
\[
y_{2,5max} = y_{1,6min}
.\] 
og der isoleres for $\lambda_2$
\end{solution}
\end{document}
