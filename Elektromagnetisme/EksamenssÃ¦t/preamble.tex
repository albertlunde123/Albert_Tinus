%%%%%%%%%%%%%%%%%%%%
%% SUPER PREAMBLE %%
%%%%%%%%%%%%%%%%%%%%

\documentclass[a4paper]{article}
\usepackage[utf8]{inputenc}
\usepackage[T1]{fontenc} % Fonts and stuff
\usepackage{amsmath, amsfonts, mathtools, amsthm, amssymb} % math

\usepackage{fancyhdr} % Header, Footer etc.
\usepackage{adforn}
\usepackage{efbox}
\usepackage{lastpage}

\pagestyle{fancy}
\fancyhf{}
\fancyhead[R]{Albert Lunde}
\fancyfoot[R]{\efbox[margin = 10pt, topline = false, leftline = false]{\thepage\ of \pageref{LastPage}}}

\renewcommand{\headrule}{%
	\hrulefill
}
\renewcommand{\footrulewidth}{0pt}

%% Margin Control %%

\def\changemargin#1#2{\list{}{\rightmargin#2\leftmargin#1}\item[]}
\let\endchangemargin=\endlist


%%%%%%%%%%%%%%%%%%%%%%%%%%%%%%%%%%%%%%%%%%%%%%%%%%%%%%%%%%

% figure support

\usepackage{import}
\usepackage{pdfpages}
\usepackage{transparent}
\usepackage[dvipsnames]{xcolor}

\newcommand{\incfig}[2][1]{%
    \def\svgwidth{#1\columnwidth}
    \import{figures/}{#2.pdf_tex}
}

\pdfsuppresswarningpagegroup=1

%%%%%%%%%%%%%%%%%%%%%%%%%%%%%%%%%%%%%%%%%%%%%%%%%%%%%%%%%%

\usepackage{tikzsymbols} % Symbols
\usepackage[framemethod=TikZ]{mdframed} % Boxes around theorem environments
\usepackage{thmtools}


% Exercise environment with surrounding box

\declaretheoremstyle[
name=\textcolor{green!50!black}{Exercise},
    postheadspace = \newline,
    spacebelow = 10pt,
    mdframed={
  backgroundcolor = green!15,
  linecolor = green!50!black,
  linewidth = 1pt,
  rightline =  false,
  topline = false,
  bottomline = false,
  skipabove=0pt,
  skipbelow=20pt,
  innerleftmargin=15pt,
  innertopmargin=10pt,
  innerrightmargin=15pt,
  innerbottommargin=10pt}
]{exercise}
\declaretheorem[style=exercise,numbered=no]{exercise}

\declaretheoremstyle[
name=\textcolor{pink!60!purple}{Subexercise},
    postheadspace = \newline,
    spacebelow = 10pt,
    mdframed={
  backgroundcolor = pink!25,
  linecolor = pink!60!purple,
  linewidth = 1pt,
  rightline =  false,
  topline = false,
  bottomline = false,
  skipabove=0pt,
  skipbelow=20pt,
  innerleftmargin=15pt,
  innertopmargin=10pt,
  innerrightmargin=15pt,
  innerbottommargin=10pt}
]{subexercise}

\declaretheoremstyle[
    name=Note,
    postheadspace = \newline,
    postheadhook={\textcolor{black}{\rule[.6ex]{\linewidth}{1pt}}\\},
    spacebelow = 10pt,
    mdframed={
  backgroundcolor = yellow!10,
  skipabove=0pt,
  skipbelow=20pt,
  innerleftmargin=15pt,
  innertopmargin=10pt,
  innerrightmargin=15pt,
  innerbottommargin=10pt}
]{margin}
\declaretheorem[style=subexercise,numbered=no]{subexercise}

% Solution environment, with coffee cup

% \newenvironment{solution}
%  {\renewcommand\qedsymbol{\tikzsymbolsuse{Coffeecup}}\begin{proof}[Solution]}
 % {\end{proof}}

\newenvironment{solution}
{%
	\textbf{\textcolor{gray}{Solution:}}
}%
{%a
	\hfill\textbf{\textcolor{red}{\tikzsymbolsuse{Coffeecup}}}
}%

% Subexercise enviroment
%  \newenvironment{subexercise}[1]
%  {
% 	\begin{changemargin}{1.0cm}{1.0cm}
% 	\textbf{(#1)}\em
% 	}{
	% \end{changemargin}
	% } 

 % \newenvironment{subexercise}[1]
 % {\noindent
	 % \textbf{(#1)} \quad \adforn{10} \quad \em
 % }{}

% Mathematical typesetting stuff.

 \newcommand{\dd}{\mathrm{\textbf{d}}}

 % Change font

\usepackage{tgadventor}
\usepackage{cmbright}
\usepackage{bm}





