\usepackage[utf8]{inputenc}
\usepackage[T1]{fontenc}
\usepackage{textcomp}

\usepackage{url}

\usepackage[
    sorting=nyt,
    style=alphabetic
]{biblatex}
\addbibresource{bibliography.bib}

\usepackage{hyperref}
\hypersetup{
    colorlinks,
    linkcolor={black},
    citecolor={black},
    urlcolor={blue!80!black}
}
\usepackage[noabbrev]{cleveref}

% Adds Bibliography, ... to Table of Contents
\usepackage[nottoc]{tocbibind}

\usepackage{graphicx}
\usepackage{float}
\usepackage[usenames,dvipsnames,svgnames]{xcolor}

% \usepackage{cmbright}

\usepackage{amsmath, amsfonts, mathtools, amsthm, amssymb}
\usepackage{mathrsfs}
\usepackage{cancel}

\newcommand\N{\ensuremath{\mathbb{N}}}
\newcommand\R{\ensuremath{\mathbb{R}}}
\newcommand\Z{\ensuremath{\mathbb{Z}}}
\renewcommand\O{\ensuremath{\emptyset}}
\newcommand\Q{\ensuremath{\mathbb{Q}}}
\newcommand\C{\ensuremath{\mathbb{C}}}
\let\implies\Rightarrow
\let\impliedby\Leftarrow
\let\iff\Leftrightarrow
\let\epsilon\varepsilon

\usepackage{tikz}
\usepackage{tikz-cd}

% theorems
\usepackage{thmtools}
\usepackage{thm-restate}
\usepackage[framemethod=TikZ]{mdframed}
\mdfsetup{skipabove=1em,skipbelow=0em, innertopmargin=12pt, innerbottommargin=8pt}

\theoremstyle{definition}

\makeatletter

\declaretheoremstyle[headfont=\bfseries\sffamily, bodyfont=\normalfont, mdframed={ nobreak } ]{thmgreenbox}
\declaretheoremstyle[headfont=\bfseries\sffamily, bodyfont=\normalfont, mdframed={ nobreak } ]{thmredbox}
\declaretheoremstyle[headfont=\bfseries\sffamily, bodyfont=\normalfont]{thmbluebox}
\declaretheoremstyle[headfont=\bfseries\sffamily, bodyfont=\normalfont]{thmblueline}
\declaretheoremstyle[headfont=\bfseries\sffamily, bodyfont=\normalfont, numbered=no, mdframed={ rightline=false, topline=false, bottomline=false, }, qed=\qedsymbol ]{thmproofbox}
\declaretheoremstyle[headfont=\bfseries\sffamily, bodyfont=\normalfont, numbered=no, mdframed={ nobreak, rightline=false, topline=false, bottomline=false } ]{thmexplanationbox}

\declaretheoremstyle[headfont=\bfseries\sffamily, bodyfont=\normalfont, numbered=no, mdframed={ nobreak, rightline=false, topline=false, bottomline=false } ]{thmexplanationbox}


\declaretheorem[numberwithin=chapter, style=thmgreenbox, name=Definition]{definition}
\declaretheorem[sibling=definition, style=thmredbox, name=Corollary]{corollary}
\declaretheorem[sibling=definition, style=thmredbox, name=Proposition]{prop}
\declaretheorem[sibling=definition, style=thmredbox, name=Theorem]{theorem}
\declaretheorem[sibling=definition, style=thmredbox, name=Lemma]{lemma}
\declaretheorem[sibling=definition, style=thmbluebox,  name=Example]{eg}
\declaretheorem[sibling=definition, style=thmbluebox,  name=Nonexample]{noneg}
\declaretheorem[sibling=definition, style=thmblueline, name=Remark]{remark}




\declaretheorem[numbered=no, style=thmexplanationbox, name=Proof]{explanation}
\declaretheorem[numbered=no, style=thmproofbox, name=Proof]{replacementproof}
\declaretheorem[style=thmbluebox,  numbered=no, name=Exercise]{ex}
\declaretheorem[style=thmblueline, numbered=no, name=Note]{note}

% \renewenvironment{proof}[1][\proofname]{\begin{replacementproof}}{\end{replacementproof}}

% \AtEndEnvironment{eg}{\null\hfill$\diamond$}%

\newtheorem*{uovt}{UOVT}
\newtheorem*{notation}{Notation}
\newtheorem*{previouslyseen}{As previously seen}
\newtheorem*{problem}{Problem}
\newtheorem*{observe}{Observe}
\newtheorem*{property}{Property}
\newtheorem*{intuition}{Intuition}


\declaretheoremstyle[
    headfont=\bfseries\sffamily\color{RawSienna!70!black}, bodyfont=\normalfont,
    mdframed={
        linewidth=2pt,
        rightline=false, topline=false, bottomline=false,
        linecolor=RawSienna, backgroundcolor=RawSienna!5,
    }
]{todo}
\declaretheorem[numbered=no, style=todo, name=TODO]{TODO}


\usepackage{etoolbox}
\AtEndEnvironment{vb}{\null\hfill$\diamond$}%
\AtEndEnvironment{intermezzo}{\null\hfill$\diamond$}%

% http://tex.stackexchange.com/questions/22119/how-can-i-change-the-spacing-before-theorems-with-amsthm
% \def\thm@space@setup{%
%   \thm@preskip=\parskip \thm@postskip=0pt
% }

\usepackage{xifthen}

\makeatother

% figure support (https://castel.dev/post/lecture-notes-2)
\usepackage{import}
\usepackage{xifthen}
\pdfminorversion=7
\usepackage{pdfpages}
\usepackage{transparent}


\makeatletter
\newif\ifworking
\@ifclasswith{tuftebook}{working}{\workingtrue}{\workingfalse}
\makeatother

\newcommand{\incfig}[2][1]{%
    % \ifworking{\makebox[0pt][c]{\color{gray}{\scriptsize\textsf{#2}}}}\fi%
    \def\svgwidth{#1\textwidth}
    \import{./figures/}{#2.pdf_tex}
}

\newcommand{\fullwidthincfig}[2][0.90]{%
    % \ifworking{\makebox[0pt][l]{\color{gray}{\scriptsize\textsf{#2}}}}\fi%
    \def\svgwidth{#1\paperwidth}
    \import{./figures/}{#2.pdf_tex}
}



\newcommand{\minifig}[2]{%
    \def\svgwidth{#1}%
    \begingroup%
    \setbox0=\hbox{\import{./figures/}{#2.pdf_tex}}%
    \parbox{\wd0}{\box0}\endgroup%
    \hspace*{0.2cm}
}

% %http://tex.stackexchange.com/questions/76273/multiple-pdfs-with-page-group-included-in-a-single-page-warning
\pdfsuppresswarningpagegroup=1

\newcommand\todo[1]{\ifworking {{\color{red}{#1}}} \else {}\fi}
\newcommand\charlotte[1]{\ifworking {{\color{blue}{#1}}} \else {}\fi}

\author{Gilles Castel}



\usepackage{multirow}
\def\block(#1,#2)#3{\multicolumn{#2}{c}{\multirow{#1}{*}{$ #3 $}}}

% \overfullrule=1mm

\newenvironment{myproof}[1][\proofname]{%
  \proof[\rm \bf #1]%
}{\endproof}

\title{\vspace{-1cm}\efbox[margin = 15pt]{August - 2019}\vspace{-1cm}}
\author{}
\date{}

\begin{document}
\pagecolor{color1}
\maketitle
\thispagestyle{fancy}
\begin{exercise}[Opgave 1]
Vi betragter det elektrostatiske arrangement vist på Figur 1. En kugleskal
med indre radius $a$ og ydre radius $b$ bærer en sfærisk symmetrisk ladnings-fordeling, hvis volumensladningstæthed er givet ved.
\[
\rho\left( r \right) = \begin{cases}
	0 \quad &\left( r<a \right) \\
	\alpha r^2 \quad &\left( a<r<b \right) \\
	0 \quad &\left( r > b \right) 
\end{cases}
.\]
hvor $r$ er afstanden fra origo er $\alpha $ er en positiv konstant. Den dielektriske permittivitet lig $\epsilon_0$ overalt.
\end{exercise}
\begin{figure}[ht]
    \centering
    \incfig[0.4]{opgave1}
    \label{fig:opgave1}
\end{figure}
\begin{subexercise}[a]
Bestem den totale ladning Q båret af kugleskallen. Bestem retningen og størrelsen af det elektriske felt i områderne $r<a$,  $a<r<b$ og $r>b$.
\end{subexercise}
\begin{solution}

\end{solution}
\begin{subexercise}[b]
Angiv det elektriske potential i området $r>b$.
Bestem det arbejde der kræves for at flytte en punktladning $q\left( q>0 \right) $
langsomt fra uendelig langt væk $\left( r = \infty \right) $ til den ydre skal $\left( r=b \right) $.
\end{subexercise}
\begin{solution}

\end{solution}
\begin{exercise}[Opgave 2]
Vi betragter det elektriske kredsløb vist på Figur 2. Kredsløbet består af
en ideel emf kilde $\mathcal{E} $, to spoler med selvinduktans $L$ og tre modstande med
modstand $R$, som indikeret på figuren. Efter at have været åben i lang tid
sluttes kontakten til tiden $t=0$. Vi ser bort fra fælles induktans effekter.
\end{exercise}
\begin{figure}[ht]
    \centering
    \incfig[0.5]{opgave2}
    \label{fig:opgave2}
\end{figure}
\begin{subexercise}[a]
Bestem værdien af strømmen $i$, som løber i kredsløbet,
 \begin{itemize}
	\item til tiden $t = 0$, lige efter kontakten sluttes
	\item til tiden $t\to \infty$
\end{itemize}
\end{subexercise}
\begin{solution}

\end{solution}
\begin{exercise}[Opgave 3]
Punktladninger med ladning $q>0$ bevæger sig med konstant hastighed $\vec{\mathbf{v}} $
langs symmetriaksen (z-aksen) af en uendelig lang metalcylinder med radius $R$, som vist på figuren. Ladningernes volumentæthed $n\left( r \right) $ er givet ved
\[
n\left( r \right) = 
\begin{cases}
	n_0\left( 1 - r /R \right) \quad & \left( r<R \right) \\
	0 \quad & \left( r > R \right) 
\end{cases}
,\] 
Hvor $n_0$ er en positiv konstant og $r$ er afstand til $z$-aksen. Den magnetiske permeabilitet er $\mu_0$ overalt.
\end{exercise}
\begin{figure}[ht]
    \centering
    \incfig{opgave3}
    \label{fig:opgave3}
\end{figure}
\begin{subexercise}[a]
Angiv strømtætheden $\vec{\mathbf{J}} $ og bestem den strøm $I$ som løber gennem cylinderen.\\
Bestem retningen og størrelsen af det magnetiske felt i områderne $r>R$ og  $r<R$.
\end{subexercise}
\begin{solution}
\end{solution}
\begin{exercise}[Opgave 4]
En stationært halvbue-formet sløjfe med radius a ligger i (xy)-planen, som vist på Figur 4, og er påsat et uniformt magnetisk felt $\vec{\mathbf{B}}_0=B_0\left(\vec{\mathbf{j}} + \vec{\mathbf{k}}  \right)/\sqrt{2} $
hvor $B_0$ er en positiv konstant. Der løber ingen strøm i sløjfen til tiden $t<0$.
Efter $t=0$ aftager det eksterne magnetiske felts størrelse exponentialt, ifølge
\[
B_0\left( t \right)  = B_0e^{-t /\tau} \quad \left( t\ge 0 \right) 
,\]
hvor $\tau$ er en positiv konstant. Sløjfens modstand er R og vi ser bort fra dens selvinduktans. Den magnetiske permeabilitet er lig $\mu_0$ overalt
\end{exercise}
\begin{figure}[ht]
    \centering
    \incfig[0.6]{opgave4}
    \label{fig:opgave4}
\end{figure}
\begin{subexercise}[a]
Bestem retningen og størrelsen af den inducerede strøm i sløjfen til tiden $t\ge 0$.
\end{subexercise}
\begin{solution}

\end{solution}
\begin{exercise}[Opgave 5]
E-feltets størrelse og retning i en elektromagnetisk planbølge der udbreder
sig i vacuum er givet ved $\vec{\mathbf{E}} \left( x,t \right) = \vec{\mathbf{j}} E_i \cos\left( kx - \omega t \right) $. Der indsættes nu et
dielektrika med brydningsindex $n=2$ i den halvdel af rummet der er givet
ved $x<0$. $\mu = \mu_0$ i hele rummet.
\end{exercise}
\begin{subexercise}[a]
En del af den indkommende bølge reflekteres ved overgangen mellem vac-
uum og dielektrikaet som angivet i formel 35.16. Bestem amplituderne
af det elektriske og magnetiske felt i den reflekterede bølge, $E_r$ og $B_r$,
udtrykt ved $E_i$ . Bestem intensiteten af den indkommende og den reflek-
terede bølge og vis at intensiteten af den totale elektromagnetiske bølge
for $x<o$ er $I_{tot} = \frac{1}{2}\epsilon_0c \frac{8}{9}E_i^2$ .
(b) Opskriv udtryk for det elektriske og magnetiske felt for den transmit
\end{subexercise}
\begin{solution}

\end{solution}
\end{document}
