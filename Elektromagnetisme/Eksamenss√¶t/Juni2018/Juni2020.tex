\usepackage[utf8]{inputenc}
\usepackage[T1]{fontenc}
\usepackage{textcomp}

\usepackage{url}

\usepackage[
    sorting=nyt,
    style=alphabetic
]{biblatex}
\addbibresource{bibliography.bib}

\usepackage{hyperref}
\hypersetup{
    colorlinks,
    linkcolor={black},
    citecolor={black},
    urlcolor={blue!80!black}
}
\usepackage[noabbrev]{cleveref}

% Adds Bibliography, ... to Table of Contents
\usepackage[nottoc]{tocbibind}

\usepackage{graphicx}
\usepackage{float}
\usepackage[usenames,dvipsnames,svgnames]{xcolor}

% \usepackage{cmbright}

\usepackage{amsmath, amsfonts, mathtools, amsthm, amssymb}
\usepackage{mathrsfs}
\usepackage{cancel}

\newcommand\N{\ensuremath{\mathbb{N}}}
\newcommand\R{\ensuremath{\mathbb{R}}}
\newcommand\Z{\ensuremath{\mathbb{Z}}}
\renewcommand\O{\ensuremath{\emptyset}}
\newcommand\Q{\ensuremath{\mathbb{Q}}}
\newcommand\C{\ensuremath{\mathbb{C}}}
\let\implies\Rightarrow
\let\impliedby\Leftarrow
\let\iff\Leftrightarrow
\let\epsilon\varepsilon

\usepackage{tikz}
\usepackage{tikz-cd}

% theorems
\usepackage{thmtools}
\usepackage{thm-restate}
\usepackage[framemethod=TikZ]{mdframed}
\mdfsetup{skipabove=1em,skipbelow=0em, innertopmargin=12pt, innerbottommargin=8pt}

\theoremstyle{definition}

\makeatletter

\declaretheoremstyle[headfont=\bfseries\sffamily, bodyfont=\normalfont, mdframed={ nobreak } ]{thmgreenbox}
\declaretheoremstyle[headfont=\bfseries\sffamily, bodyfont=\normalfont, mdframed={ nobreak } ]{thmredbox}
\declaretheoremstyle[headfont=\bfseries\sffamily, bodyfont=\normalfont]{thmbluebox}
\declaretheoremstyle[headfont=\bfseries\sffamily, bodyfont=\normalfont]{thmblueline}
\declaretheoremstyle[headfont=\bfseries\sffamily, bodyfont=\normalfont, numbered=no, mdframed={ rightline=false, topline=false, bottomline=false, }, qed=\qedsymbol ]{thmproofbox}
\declaretheoremstyle[headfont=\bfseries\sffamily, bodyfont=\normalfont, numbered=no, mdframed={ nobreak, rightline=false, topline=false, bottomline=false } ]{thmexplanationbox}

\declaretheoremstyle[headfont=\bfseries\sffamily, bodyfont=\normalfont, numbered=no, mdframed={ nobreak, rightline=false, topline=false, bottomline=false } ]{thmexplanationbox}


\declaretheorem[numberwithin=chapter, style=thmgreenbox, name=Definition]{definition}
\declaretheorem[sibling=definition, style=thmredbox, name=Corollary]{corollary}
\declaretheorem[sibling=definition, style=thmredbox, name=Proposition]{prop}
\declaretheorem[sibling=definition, style=thmredbox, name=Theorem]{theorem}
\declaretheorem[sibling=definition, style=thmredbox, name=Lemma]{lemma}
\declaretheorem[sibling=definition, style=thmbluebox,  name=Example]{eg}
\declaretheorem[sibling=definition, style=thmbluebox,  name=Nonexample]{noneg}
\declaretheorem[sibling=definition, style=thmblueline, name=Remark]{remark}




\declaretheorem[numbered=no, style=thmexplanationbox, name=Proof]{explanation}
\declaretheorem[numbered=no, style=thmproofbox, name=Proof]{replacementproof}
\declaretheorem[style=thmbluebox,  numbered=no, name=Exercise]{ex}
\declaretheorem[style=thmblueline, numbered=no, name=Note]{note}

% \renewenvironment{proof}[1][\proofname]{\begin{replacementproof}}{\end{replacementproof}}

% \AtEndEnvironment{eg}{\null\hfill$\diamond$}%

\newtheorem*{uovt}{UOVT}
\newtheorem*{notation}{Notation}
\newtheorem*{previouslyseen}{As previously seen}
\newtheorem*{problem}{Problem}
\newtheorem*{observe}{Observe}
\newtheorem*{property}{Property}
\newtheorem*{intuition}{Intuition}


\declaretheoremstyle[
    headfont=\bfseries\sffamily\color{RawSienna!70!black}, bodyfont=\normalfont,
    mdframed={
        linewidth=2pt,
        rightline=false, topline=false, bottomline=false,
        linecolor=RawSienna, backgroundcolor=RawSienna!5,
    }
]{todo}
\declaretheorem[numbered=no, style=todo, name=TODO]{TODO}


\usepackage{etoolbox}
\AtEndEnvironment{vb}{\null\hfill$\diamond$}%
\AtEndEnvironment{intermezzo}{\null\hfill$\diamond$}%

% http://tex.stackexchange.com/questions/22119/how-can-i-change-the-spacing-before-theorems-with-amsthm
% \def\thm@space@setup{%
%   \thm@preskip=\parskip \thm@postskip=0pt
% }

\usepackage{xifthen}

\makeatother

% figure support (https://castel.dev/post/lecture-notes-2)
\usepackage{import}
\usepackage{xifthen}
\pdfminorversion=7
\usepackage{pdfpages}
\usepackage{transparent}


\makeatletter
\newif\ifworking
\@ifclasswith{tuftebook}{working}{\workingtrue}{\workingfalse}
\makeatother

\newcommand{\incfig}[2][1]{%
    % \ifworking{\makebox[0pt][c]{\color{gray}{\scriptsize\textsf{#2}}}}\fi%
    \def\svgwidth{#1\textwidth}
    \import{./figures/}{#2.pdf_tex}
}

\newcommand{\fullwidthincfig}[2][0.90]{%
    % \ifworking{\makebox[0pt][l]{\color{gray}{\scriptsize\textsf{#2}}}}\fi%
    \def\svgwidth{#1\paperwidth}
    \import{./figures/}{#2.pdf_tex}
}



\newcommand{\minifig}[2]{%
    \def\svgwidth{#1}%
    \begingroup%
    \setbox0=\hbox{\import{./figures/}{#2.pdf_tex}}%
    \parbox{\wd0}{\box0}\endgroup%
    \hspace*{0.2cm}
}

% %http://tex.stackexchange.com/questions/76273/multiple-pdfs-with-page-group-included-in-a-single-page-warning
\pdfsuppresswarningpagegroup=1

\newcommand\todo[1]{\ifworking {{\color{red}{#1}}} \else {}\fi}
\newcommand\charlotte[1]{\ifworking {{\color{blue}{#1}}} \else {}\fi}

\author{Gilles Castel}



\usepackage{multirow}
\def\block(#1,#2)#3{\multicolumn{#2}{c}{\multirow{#1}{*}{$ #3 $}}}

% \overfullrule=1mm

\newenvironment{myproof}[1][\proofname]{%
  \proof[\rm \bf #1]%
}{\endproof}

\title{\vspace{-1cm}\efbox[margin = 15pt]{\textbf{Juni - 2018}}\vspace{-1cm}}
\author{}
\date{}

\begin{document}
\maketitle
\thispagestyle{fancy}
\begin{exercise}[Opgave 1]
Vi betragter det elektrostatiske arrangement vist på Figur 1. En kugle med radius $a$ og centreret i origo er ladet med en sfærisk symmetrisk volumensladningstæthed $rho\left( r \right) = \alpha\, r\left( r<a \right) $, hvor $a$ er en positiv konstant og r afstanden til origo. Den totale ladning båret af kuglen er $Q$. En sfærisk metalskal, placeret koncentrisk med kuglen og med radius $b$, bærer en total ladning $-Q$. Der er vakuum i områderne a < r < b og r > b, og den dielektriske
permittivitet er lig $\epsilon_0$ overalt.
\end{exercise}
\begin{figure}[ht]
    \centering
    \incfig[0.4]{opgave1}
\end{figure}
\begin{subexercise}[a]
Udtryk $Q$ som funktion af $\alpha $.\\
Bestem retningen og størrelsen af det eletriske felt i områderne $r<a$,  $a<r<b$ og $r>b$.
\end{subexercise}
\begin{solution}

\end{solution}
\begin{subexercise}[b]
Besyrm den elektriske potentialforskel mellem kuglens overflade $\left( r=a \right) $ og metalskallen $\left( r=b \right) $, og udtryk kacapacitansen af den kapacitor dannet af disse elektroder som funktion af $\epsilon_0$ og dimensionerne i problemet. Bestem endvidere hele ladningsfordelingens totale elektriske potentielle energi.
\end{subexercise}
\begin{solution}

\end{solution}
\begin{exercise}[Opgave 2]
Vi betragter det elektriske kredsløb vist på Figur 2. Kredsløbet består af en ideel emf kilde $\mathcal{E} $, en kapacitor med  kapacitans $C$ og fire modstande med modstand $R$ eller $2R$ som indikeret på figuren. Til tiden $t=0$ er kapacitoren afladt og kontakten sluttes.
\end{exercise}
\begin{figure}[ht]
    \centering
    \incfig{opgave2}
    \caption{Opgave2}
    \label{fig:opgave2}
\end{figure}
\begin{subexercise}[a]
Bestem værdien af strømmen $i$, som løber i kredsløbet
\begin{itemize}
	\item til tiden $t=0$, lige efter kontakten sluttes
	\item til tiden $t\to \infty$
\end{itemize}
Bestem endvidere den elektriske potentialforskel $v_{ab}$ mellem enderne af kapacitoren når $t\to \infty$.
\end{subexercise}
\begin{solution}

\end{solution}
\begin{exercise}[Opgave 3]
En friktionsløs metalstang bevæger sig med konstant hastighed $v$ langs z-aksen på to parallelle skinner med indbyrdes afstand $L$, som vist på Figur 3. En uendelig lang ledning, hvori der løber en konstant strøm $I$, ligger langs z-aksen i en afstand $a$ af skinnen til venstre. Stangens modstand er $R$ og vi ser bort fra skinnernes modstand, såvel som selvinduktans effekter og tyngdekraften. Den magnetiske permeabilitet er lig $\mu_0$ overalt.
\end{exercise}
\begin{figure}[ht]
    \centering
    \incfig{opgave3}
    \caption{Opgave3}
    \label{fig:opgave3}
\end{figure}
\begin{subexercise}[a]
Bestem retningen og størrelsen af den inducerede strøm i det kredsløb dannet af metalstangen og skinnerne.
\end{subexercise}
\begin{solution}

\end{solution}
\begin{subexercise}[b]
En operatør påvirker stangen med en kraft, som får den til at bevæge sig med konstant hastighed langs z-aksen. Bestem effekten af det arbejde som operatøren skal udføre, såvel som Joule effekten i modstanden.
\end{subexercise}
\begin{solution}

\end{solution}
\begin{exercise}[Opgave 5]
Kohærent lys med bølgelængden $\lambda$ sendes mod en skærm med to spalter med bredden $d /2$, i en indbyrdes afstand $d$ (målt fra center til center af de to spalter) som angivet på Figur 5. I afstanden $R$ fra skærmen med spalterne er anbragt en observationsskærm. $y$ angiver højden på observationsskærmen i forhold til midterlinjen ved $y=0$. Det antages at $R \gg y \gg d$ og at $\lambda < d /2}$.
\end{exercise}
\begin{subexercise}[a]
Opskriv et udtryk for intensiteten, $I\left( y \right) $, af det resulterende diffraktionsmønster på skærmen som funktion af $y$. Udtryk intensiteten ved $d$, $y$, $R$, $\lambda$, $I_0$, hvor $I_0= I\left( 0 \right) $ er intensiteten ved $y=0$.\\
Skitser intensiteten som funktion af $y$ (medtag de første 5 minima på hver side af $y = 0$. Vær specielt opmærksom på at angive korrekte positioner for $y$-værdier der resulterer i minimal intensitet.
\end{subexercise}
\begin{figure}[ht]
    \centering
    \incfig{opgave5}
    \caption{Opgave5}
    \label{fig:opgave5}
\end{figure}
\begin{solution}

\end{solution}
\end{document}
