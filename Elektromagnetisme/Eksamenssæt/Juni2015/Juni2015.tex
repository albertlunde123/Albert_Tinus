%%%%%%%%%%%%%%%%%%%%
%% SUPER PREAMBLE %%
%%%%%%%%%%%%%%%%%%%%

\documentclass[a4paper]{article}
\usepackage[utf8]{inputenc}
\usepackage[T1]{fontenc} % Fonts and stuff
\usepackage{amsmath, amsfonts, mathtools, amsthm, amssymb} % math

\usepackage{fancyhdr} % Header, Footer etc.
\usepackage{adforn}

\pagestyle{fancy}
\fancyhf{}
\fancyhead[R]{Albert Lunde, John Faxe Jensen}
\fancyfoot[C]{\adforn{17}\quad\thepage\quad\adforn{45}}

\renewcommand{\headrule}{%
	\hrulefill
	 {\quad\adforn{21}\adforn{10}\adforn{49}\quad}%
	\hrulefill}
\renewcommand{\footrulewidth}{0pt}

%% Margin Control %%

\def\changemargin#1#2{\list{}{\rightmargin#2\leftmargin#1}\item[]}
\let\endchangemargin=\endlist


%%%%%%%%%%%%%%%%%%%%%%%%%%%%%%%%%%%%%%%%%%%%%%%%%%%%%%%%%%

% figure support

\usepackage{import}
\usepackage{pdfpages}
\usepackage{transparent}
\usepackage{xcolor}

\newcommand{\incfig}[2][1]{%
    \def\svgwidth{#1\columnwidth}
    \import{figures/}{#2.pdf_tex}
}

\pdfsuppresswarningpagegroup=1

%%%%%%%%%%%%%%%%%%%%%%%%%%%%%%%%%%%%%%%%%%%%%%%%%%%%%%%%%%

\usepackage{tikzsymbols} % Symbols
\usepackage{mdframed} % Boxes around theorem environments
\usepackage{thmtools}


% Exercise environment with surrounding box

\declaretheoremstyle[
    name=Exercise,
    mdframed={
  skipabove=0pt,
  skipbelow=20pt,
  innerleftmargin=10pt,
  innerrightmargin=10pt,
  innerbottommargin=7pt}
]{thmsty}
\declaretheorem[style=thmsty,numbered=no]{exercise}

% Solution environment, with coffee cup

\newenvironment{solution}
 {\renewcommand\qedsymbol{\tikzsymbolsuse{Coffeecup}}\begin{proof}[Solution]}
 {\end{proof}}

% Subexercise enviroment
%  \newenvironment{subexercise}[1]
%  {
% 	\begin{changemargin}{1.0cm}{1.0cm}
% 	\textbf{(#1)}\em
% 	}{
	% \end{changemargin}
	% } 

 \newenvironment{subexercise}[1]
 {\noindent
	 \textbf{(#1)} \quad \adforn{10} \quad \em
 }{}

% Mathematical typesetting stuff.

 \newcommand{\dd}{\mathrm{\textbf{d}}}



\title{\vspace{-1cm}\efbox[margin = 15pt]{\textbf{Juni - 2015}}\vspace{-1cm}}
\author{}
\date{}

\begin{document}
\pagecolor{color1}
\maketitle
\thispagestyle{fancy}
\begin{exercise}[Opgave 1]
	Vi betragter to koncentriske metalkugleskaller i vakuum, som vist på figuren. Den indre skal $\left( a_1 < r <a_2 \right) $ bærer en ladning $Q_1$ og den ydre skal bærer en ladning $Q_2$. Vi antager at den dielektriske konstant $K$ er lig $1$ overalt, hvor der ikke er metal.
\end{exercise}
\begin{figure}[ht]
    \centering
    \incfig{opgave1}
    \label{fig:opgave1}
\end{figure}
\begin{subexercise}[a]
Bestem størrelse og retning af det elektriske felt i områderne $r<a_1$, $a_1<r<a_2$, $a_2<r<b_1$, $b_1<r<b_2$ og $r>b_2$
\end{subexercise}
\begin{solution}
	\begin{itemize}
		\item[$r<a_1$:]
			Indenfor for skallerne er det elektriske felt altid $0$.
		\item[$a_1<r<a_2$:]
			Det elektriske felt er altid $0$ inde i en konduktor.
		\item[$a_2<r<b_1$:]
			Udenfor skallen, kan denne betragtes som en punktladning centreret i origo med ladning $Q_1$. Det elektriske bliver således,
			\[
				\vec{\mathbf{E}} (r) = \frac{1}{4\pi\epsilon_0}\frac{Q_1}{r^2}\hat{r}
			.\] 
		\item[$b_1<r<b_2$:]
			Det elektriske felt er altid $0$ inde i en konduktor.
		\item[$r>b_2$:]
			Nu påvirker begge skaller det elektriske felt. Så det elektriske felt bliver,
			\[
				\vec{\mathbf{E}} (r) = \frac{1}{4\pi\epsilon_0}\frac{Q_1+Q_2}{r^2}\hat{r}
			.\] 
	\end{itemize}
\end{solution}
\begin{subexercise}[b]
	Bestem overfladeladningstæthederne på de fire overflader.
\end{subexercise}
\begin{solution}
Hvis jeg ligger en gauss-sfære med overflade i midten af den inderste skal, skal den totale indeholdne ladning være $0$. Dette medfører at,
 \[
\sigma_{a_1} = 0
.\] 
Den resterende ladning $Q_1$ må derfor sidde på den ydre side. Dermed,
\[
\sigma_{a_2} = \frac{Q_1}{4\pi a_2^2}
.\]
Hvis jeg nu lægger min gauss-sfære så dennes overflade ligger i midten af den ydre skal, må der igen gælde at den totale indeholdne ladning er 0. Der skal derfor ligge $-Q_1$ på den indre side.
\[
\sigma_{b_1} = \frac{-Q_1}{4\pi b_1^2}
.\]
Den resterende ladning $Q_2 - (-Q_1) = Q_2+Q_1$ må derfor ligge på den ydre side.
\[
\sigma_{b_2} = \frac{Q_2+Q_1}{4\pi b_{2}^2}
.\] 
\end{solution}
\begin{exercise}[Opgave 2]
	Vi betragter to koaksiale cylinderer  som vist ooå figuren. Der løber konstante positive strøne $I_1$ og $I_2$ i områderne $0<r<a$ og $b<r<c$, henholdsvis. Der er vakuum i områderne  $a<r<b$ og $r>c$, og den relative magnetiske permittivitet $K_m$ er lig 1 overalt. Strømtætheden er givet som,
	\[
	J\left( r \right) = \begin{cases}
		-\alpha r \quad &\left( 0<r<a \right) \\
		0 \quad &\left( a<r<b \right) \\
		\alpha r \quad &\left( b<r<c \right) \\ 
		0 \quad &\left(r>c \right) 
	\end{cases}
	.\]
	Hvor $\alpha $ er en positiv konstant.
\end{exercise}
\begin{figure}[ht]
    \centering
    \incfig{opgave2}
    \label{fig:opgave2}
\end{figure}
\begin{subexercise}[a]
Bestem $I_1$ og $I_2$ udtrykt ved konstant $\alpha $ og dimensionerne $a$, $b$ og $c$.
\end{subexercise}
\begin{solution}
Jeg betragter infinitesimale cirkelstykke, da jeg ved at strømtætheden gennem disse er konstant. Jeg beregner gennem et vilkårligt element og integrerer. Strømmen gennem et cirkelstykke i en afstand $r$ er lig,
 \[
dI\left( r \right) = J\left( r \right) 2\pi r \dd r
.\] 
Nu beregner jeg $I_1$ ved at integrer over $0 \to a$.
\[
I_1 = \int_0^{a} \dd I = -\alpha 2\pi \int_{0}^{a} r^2\, \dd r = -\alpha 2\pi \frac{1}{3}a^{3} = -\alpha\pi \frac{2}{3} a^3
.\]
Jeg beregner $I_2$ ved at integrerer over $b \to c$.
\[
	I_2 = \alpha 2\pi\int_{b}^{c} r^2\, \dd r = \alpha\pi \frac{2}{3}\left( c^{3} - b^3 \right) 
.\] 
\end{solution}
\begin{subexercise}[b]
Bestem størrelse og retning af det magnetiske felt i områderne $0<r<a$, $a<r<b$, $b<r<c$ og $r>c$. Hvordan skal dimensioner $a $, $b$ og $c$ vælges så det magnetiske felt er nul udenfor kablet?
\end{subexercise}
\begin{solution}
Det magnetiske felt udenfor en leder er givet på følgende måde,
\[
B = \frac{\mu_0 I}{2\pi r}
.\] 
Retningen er $\vec{\mathbf{I}} \times \hat{r} = \frac{\vec{\mathbf{B}} }{\|B\|}$. Jeg skal altså blot bestemme størrelsen på strømmen i de gældende områder.
\begin{itemize}
	\item[$0<r<a$]
		I denne region er mængden af strøm,
		 \[
		I\left( -\alpha\pi\frac{2}{3}r^3 \right) 
		.\] 
		Magnetfeltet bliver da,
		\[
		B = \frac{-\mu_0 \alpha}{3} r^2
		.\]
	\item[$a<r<b$]
		I denne region er strømmen uafhængig af $r$. Magnetfeltets størrelse bliver,
		\[
		B = \frac{-\mu_0 \alpha}{3} \frac{a^3}{r}
		.\] 
	\item[$b<r<c$]
		I denne region skal jeg både tage højde for strømmen igennem den inderste cylinder og den igennem den yderste cylinder. Strømmen igennem den yderste er,
		 \[
		\alpha\pi \frac{2}{3}\left( r^{3} - b^3 \right) 
		.\] 
		Den totale er dermed,
		\[
		I_{tot} = \alpha\pi \frac{2}{3}\left( r^{3} - b^3 \right)  -\alpha\pi \frac{2}{3} a^3 = \alpha \pi \frac{2}{3}\left( r^3-b^3-a^3 \right)  
		.\] 
		$B$-feltet er dermed,
		 \[
		B = \frac{\mu_0\alpha }{3}\frac{\left( r^3-b^3-a^3 \right) }{r}
		.\] 
	\item[$r>c$]
		Her er $B$-feltet,
		\[
		B = \frac{\mu_0\alpha }{3}\frac{\left( c^3-b^3-a^3 \right) }{r}
		.\]
\end{itemize}
Hvis $B$-feltet skal være $0$, må der gælde at,
 \[
c ^3 - b^3 - a^3 = 0
.\] 
\end{solution}
\begin{exercise}[Opgave 3]
To punktladninger i vakuum, hver med en negativ ladning $-Q$, er stationære på $y$-aksen ved $y = \pm a$. En tredje punktladning med positiv ladning $q$ og masse $m$, som til tiden $t=0$ befinder sig i hvile på $x$-aksen ved $x=-5a$, er tiltrukket af de to faste ladninger. Der ses bort fra de andre kræfter.
\end{exercise}
\begin{figure}[ht]
    \centering
    \incfig[0.6]{opgave3}
    \label{fig:opgave3}
\end{figure}
\begin{subexercise}[a]
Hvad er den tredje partikels hastighed, når den passerer origo?
\end{subexercise}
\begin{solution}
	Jeg beregner den tredje partikels elektriske potentiale i dens nuværende position og ved origo. Energiforskellen vil blive omdannet til kinetisk energi. I startpositionen har partikel $q$ potentialet.
	 \[
		 U_0 = \frac{q}{4\pi\epsilon_0}\left( \frac{-2Q}{\sqrt{25a^2 + a^2} } \right) = \frac{q}{4\pi\epsilon_0}\left( \frac{-2Q}{\sqrt{26}a } \right) 
	.\]
	I slutpositionen har den potentialet,
	\[
	U_1 = \frac{q}{4\pi\epsilon_0}\left( \frac{-2Q}{a} \right) 
	.\] 
	Forskellen mellem disse størrelser, vil være tilvæksten i kinetisk energi.
	\[
	\Delta U = -\Delta K
	.\]
	Denne beregnes,
	\[
		\Delta U = U_1 - U_0 = \frac{q}{4\pi\epsilon_0}(-2Q)\left( \frac{\sqrt{26} -1 }{\sqrt{26}a }  \right) 
	.\]
	Da partiklen var i hvile inden må der også gælde at,
	\[
	\Delta K = \frac{1}{2}mv^2
	.\] 
	Jeg sætter det lig hinanden og isolerer på $v$.
	 \begin{align*}
		 \frac{1}{2}mv^2 &= \frac{qQ}{2\pi\epsilon_0a}\left( 1 - \frac{1}{\sqrt{26} } \right) \\
		v &= \sqrt{\frac{qQ}{\pi\epsilon_0am}\left( 1 - \frac{1}{\sqrt{26} } \right) }  \\
	.\end{align*}
\end{solution}\\
\begin{exercise}[Opgave 4]
Vi betragter to uendelige, stationære ledninger, som vist på Figur 4. Ledning 1 er parallel med x-aksen og krydser (yz)-planen i punktet $(0, −a, −a)$. Ledning 2 er parallel med z-aksen og krydser (xy)-planen i punktet $(a, a, 0)$. En konstant strøm I løber i ledning 1 i den negative x-retning og i ledning 2 i den positive z-retning.
\end{exercise}
\begin{figure}[ht]
    \centering
    \incfig[0.4]{opgave4}
    \label{fig:opgave4}
\end{figure}
\begin{subexercise}[a]
Vis at det magnetiske felt i origo kan skrives som $\vec{\mathbf{B}} _O = \beta I\left( \hat{i}-\hat{k} \right) $, hvor $\hat{i}$ og $\hat{k}$ er $x$- og $z$-aksers enhedsvektorer og angiv konstanten $\beta $. 
\end{subexercise}
\begin{solution}
Magnetfeltet fra den der er parallel med $z$-aksen har en $x$-komponent og $y$-komponent. På vektorform ser den ud på følgende måde.
 \[
B_1 = \frac{mu_0I}{4\pi a}\begin{pmatrix}1 \\ -1 \\ 0\end{pmatrix} 
.\]
Jeg har gjort brug af at størrelsen af magnetfeltet er $\mu_0I / 2\sqrt{2} \pi$, og at splittes op med $\sqrt{2}$.
Magnetfeltet fra den anden er.
 \[
B_1 = \frac{mu_0I}{4\pi a}\begin{pmatrix}0 \\ 1 \\ -1\end{pmatrix} 
.\].
Summen af disse er det resulterende magnetfelt
\[
\vec{\mathbf{B}}_{res} = \frac{mu_0I}{4\pi a} \begin{pmatrix}1 \\ 0 \\ -1  \end{pmatrix} = \frac{\mu_0 I}{4\pi a}\left( \hat{i}- \hat{k} \right)  
.\] 
\end{solution}
\begin{subexercise}[b]
En lille cirkulær strømkreds med areal $A$ ligger i origo i $\left( xy \right) $-planen. Vi antager, at det magnetiske felt $\vec{\mathbf{B}} _O = \beta I\left( \hat{i}-\hat{k} \right) $ er uniformt over hele strømkredsens udstrækning, da denne er meget lille. for $t>0$ reduceres strømmen i de to ledninger til $I\left( t \right) = I_0e^{-\alpha t}$, hvor $I_0$ og $\alpha $ er positive konstanter. Strømkredsens modstand er $R$.
\end{subexercise}
\begin{solution}
Jeg behøver kun at betragte den del af magnetfeltet der er vinkelret på strømkredsen. Dette er netop,
\[
B = \beta I\left( t \right)  \left( -\hat{k} \right) 
.\]
Fluxen gennem strømkredsen bliver så,
\[
\Phi_B = -\beta I\left( t \right)  A
.\] 
Nu skal jeg differentierer dette udtryk mht. $t$.
 \begin{align*}
	\frac{\dd \Phi_B}{\dd t }= \alpha \beta I_0e^{-\alpha t}A
.\end{align*}
Den inducerede strøm er dermed,
\[
I'\left( t \right) = \frac{\alpha \beta I_0e^{-\alpha t}A}{R}
.\] 
Denne strøm er med uret.
\end{solution}
\begin{subexercise}[c]
Bestem det magnetiske kraftmoment udøvet af $B$-feltet på strømkredsen og beregn strømkredsens magnetiske potentielle energi.
\end{subexercise}
\begin{solution}
Jeg starter med at beregne kraftmomentet, jeg bruger at det magnetiske dipolmoment er givet som,
\[
\vec{\mu} = I'\left( t \right) A\hat{k}
.\] 
\begin{align*}
	\tau &= \vec{\mu} \times \vec{\mathbf{B}} \\
	&= \left( I'\left( t \right) A\hat{k} \right) \times \left( \beta I\left( t \right) \left( \hat{i}-\hat{k} \right)  \right)  \\
	&= I'\left( t \right) A\hat{k} \times \beta I\left( t \right) \hat{i}\\
	&= I'\left( t \right) I\left( t \right) A\beta \hat{j} \\
.\end{align*}
Til at bestemme den potentielle energi bruger jeg,
\[
U = -\mu \cdot  \vec{\mathbf{B}} 
.\] 
\end{solution}
\begin{exercise}[Opgave 5]
Vi betragter kredsløbet vist på Figur 5, som består af en ideel emf kilde $\mathcal{E} $, to modstande, en ideel spole og tre kapacitorer.
\end{exercise}
\begin{figure}[ht]
    \centering
    \incfig{opgave5}
    \caption{Opgave5}
    \label{fig:opgave5}
\end{figure}
\begin{subexercise}[a]
Reducer kredsløbet til et ækvivalent $RLC$-kredsløb. Tegn dette og angiv $R_eq$ og $C_eq$.
\end{subexercise}
\begin{solution}
Jeg starter med at beregne de to størrelser,
\[
R_eq = \left( \frac{1}{R} + \frac{1}{R} \right) ^{-1} = \frac{R}{2}
.\] 
og kapacitansen er,
\[
C_eq = \left( \frac{1}{2C} + 1C \right)^{-1} = \frac{2}{3}C
.\]
\begin{figure}[ht]
    \centering
    \incfig[0.5]{opgave5a}
    \label{fig:opgave5a}
\end{figure}
\end{solution}
\begin{subexercise}[b]
	Kontakten sluttes ved tiden $t=0$. Strømmene $i_1$ og $i_2$ løber igennem kredsløbet som indikeret på figuren\\
	Bestem $i_1$ og $i_2$ til tiden $t=0$, lige efter kontakten sluttes. Når $t \to \infty$ bliver strømmene $i_1$ og $i_2$ konstante. Bestem deres værdier.
\end{subexercise}
\begin{solution}
Til tiden $t=0$ løber der ingen strøm igennem induktoren, da denne opfører sig som en resistor med uendelig resistans. Kapacitorerne opfører sig derimod som almindelig ledning. Strømmen $i_2$ er dermed blot lig den totale strøm i kredsløbet.
\[
	i_2 = \frac{\mathcal{E} }{R_{eq}} 
.\]
til tiden $t=\infty$ er situationen omvendt, induktoren er ledning og kapacitoren har uendelig resistans.
 \[
i_1 = \frac{\mathcal{E} }{R_{eq}}
.\] 
\end{solution}
\begin{exercise}[Opgave 6]
En kapacitor består af to parallelle plader. Pladernes udstrækning er
langt større end deres indbyrdes afstand. Pladerne befinder sig i vakuum og
kan bevæge sig uden friktion langs x-aksen. De er forbundet til to ledende
fjedre, se Figur 6. Punkterne $a$ og $b$ er stationære. Vi antager at Hookes lov
gælder for fjedrene, og at de har samme fjederkonstant k. I udgangssituatio-
nen er fjedrene i hvile, afstanden mellem pladerne er d og kapacitoren, med
kapacitans $C$, er afladt. På et tidspunkt forbindes kapacitoren til en ideel
emf kilde med elektromotorisk kraft $\mathcal{E} $ ved at slutte kontakten. Når systemet
senere er faldet til ro med kapacitoren opladt, er afstanden mellem pladerne
halveret.
\end{exercise}
\begin{figure}[ht]
    \centering
    \incfig{opgave6}
    \label{fig:opgave6}
\end{figure}
\begin{subexercise}[a]
Angiv kapacitorens kapacitans når systemet er faldet til ro, udtrykt ved
kapacitorens oprindelige kapacitans $C$. Hvad er pladernes ladning i denne
situation?
\end{subexercise}
\begin{solution}

\end{solution}
\end{document}
