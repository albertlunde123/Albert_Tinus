%%%%%%%%%%%%%%%%%%%%
%% SUPER PREAMBLE %%
%%%%%%%%%%%%%%%%%%%%

\documentclass[a4paper]{article}
\usepackage[utf8]{inputenc}
\usepackage[T1]{fontenc} % Fonts and stuff
\usepackage{amsmath, amsfonts, mathtools, amsthm, amssymb} % math

\usepackage{fancyhdr} % Header, Footer etc.
\usepackage{adforn}

\pagestyle{fancy}
\fancyhf{}
\fancyhead[R]{Albert Lunde, John Faxe Jensen}
\fancyfoot[C]{\adforn{17}\quad\thepage\quad\adforn{45}}

\renewcommand{\headrule}{%
	\hrulefill
	 {\quad\adforn{21}\adforn{10}\adforn{49}\quad}%
	\hrulefill}
\renewcommand{\footrulewidth}{0pt}

%% Margin Control %%

\def\changemargin#1#2{\list{}{\rightmargin#2\leftmargin#1}\item[]}
\let\endchangemargin=\endlist


%%%%%%%%%%%%%%%%%%%%%%%%%%%%%%%%%%%%%%%%%%%%%%%%%%%%%%%%%%

% figure support

\usepackage{import}
\usepackage{pdfpages}
\usepackage{transparent}
\usepackage{xcolor}

\newcommand{\incfig}[2][1]{%
    \def\svgwidth{#1\columnwidth}
    \import{figures/}{#2.pdf_tex}
}

\pdfsuppresswarningpagegroup=1

%%%%%%%%%%%%%%%%%%%%%%%%%%%%%%%%%%%%%%%%%%%%%%%%%%%%%%%%%%

\usepackage{tikzsymbols} % Symbols
\usepackage{mdframed} % Boxes around theorem environments
\usepackage{thmtools}


% Exercise environment with surrounding box

\declaretheoremstyle[
    name=Exercise,
    mdframed={
  skipabove=0pt,
  skipbelow=20pt,
  innerleftmargin=10pt,
  innerrightmargin=10pt,
  innerbottommargin=7pt}
]{thmsty}
\declaretheorem[style=thmsty,numbered=no]{exercise}

% Solution environment, with coffee cup

\newenvironment{solution}
 {\renewcommand\qedsymbol{\tikzsymbolsuse{Coffeecup}}\begin{proof}[Solution]}
 {\end{proof}}

% Subexercise enviroment
%  \newenvironment{subexercise}[1]
%  {
% 	\begin{changemargin}{1.0cm}{1.0cm}
% 	\textbf{(#1)}\em
% 	}{
	% \end{changemargin}
	% } 

 \newenvironment{subexercise}[1]
 {\noindent
	 \textbf{(#1)} \quad \adforn{10} \quad \em
 }{}

% Mathematical typesetting stuff.

 \newcommand{\dd}{\mathrm{\textbf{d}}}



\title{\vspace{-1cm}\efbox[margin = 15pt]{\textbf{Juni - 2018}}\vspace{-1cm}}
\author{}
\date{}

\begin{document}
\maketitle
\thispagestyle{fancy}
\begin{exercise}[Opgave 1]
Vi betragter det elektrostatiske arrangement vist på Figur 1. En kugle med radius $a$ og centreret i origo er ladet med en sfærisk symmetrisk volumensladningstæthed $rho\left( r \right) = \alpha\, r\left( r<a \right) $, hvor $a$ er en positiv konstant og r afstanden til origo. Den totale ladning båret af kuglen er $Q$. En sfærisk metalskal, placeret koncentrisk med kuglen og med radius $b$, bærer en total ladning $-Q$. Der er vakuum i områderne a < r < b og r > b, og den dielektriske
permittivitet er lig $\epsilon_0$ overalt.
\end{exercise}
\begin{figure}[ht]
    \centering
    \incfig[0.4]{opgave1}
\end{figure}
\begin{subexercise}[a]
Udtryk $Q$ som funktion af $\alpha $.\\
Bestem retningen og størrelsen af det eletriske felt i områderne $r<a$,  $a<r<b$ og $r>b$.
\end{subexercise}
\begin{solution}

\end{solution}
\begin{subexercise}[b]
Besyrm den elektriske potentialforskel mellem kuglens overflade $\left( r=a \right) $ og metalskallen $\left( r=b \right) $, og udtryk kacapacitansen af den kapacitor dannet af disse elektroder som funktion af $\epsilon_0$ og dimensionerne i problemet. Bestem endvidere hele ladningsfordelingens totale elektriske potentielle energi.
\end{subexercise}
\begin{solution}

\end{solution}
\begin{exercise}[Opgave 2]
Vi betragter det elektriske kredsløb vist på Figur 2. Kredsløbet består af en ideel emf kilde $\mathcal{E} $, en kapacitor med  kapacitans $C$ og fire modstande med modstand $R$ eller $2R$ som indikeret på figuren. Til tiden $t=0$ er kapacitoren afladt og kontakten sluttes.
\end{exercise}
\begin{figure}[ht]
    \centering
    \incfig{opgave2}
    \caption{Opgave2}
    \label{fig:opgave2}
\end{figure}
\begin{subexercise}[a]
Bestem værdien af strømmen $i$, som løber i kredsløbet
\begin{itemize}
	\item til tiden $t=0$, lige efter kontakten sluttes
	\item til tiden $t\to \infty$
\end{itemize}
Bestem endvidere den elektriske potentialforskel $v_{ab}$ mellem enderne af kapacitoren når $t\to \infty$.
\end{subexercise}
\begin{solution}

\end{solution}
\begin{exercise}[Opgave 3]
En friktionsløs metalstang bevæger sig med konstant hastighed $v$ langs z-aksen på to parallelle skinner med indbyrdes afstand $L$, som vist på Figur 3. En uendelig lang ledning, hvori der løber en konstant strøm $I$, ligger langs z-aksen i en afstand $a$ af skinnen til venstre. Stangens modstand er $R$ og vi ser bort fra skinnernes modstand, såvel som selvinduktans effekter og tyngdekraften. Den magnetiske permeabilitet er lig $\mu_0$ overalt.
\end{exercise}
\begin{figure}[ht]
    \centering
    \incfig{opgave3}
    \caption{Opgave3}
    \label{fig:opgave3}
\end{figure}
\begin{subexercise}[a]
Bestem retningen og størrelsen af den inducerede strøm i det kredsløb dannet af metalstangen og skinnerne.
\end{subexercise}
\begin{solution}

\end{solution}
\begin{subexercise}[b]
En operatør påvirker stangen med en kraft, som får den til at bevæge sig med konstant hastighed langs z-aksen. Bestem effekten af det arbejde som operatøren skal udføre, såvel som Joule effekten i modstanden.
\end{subexercise}
\begin{solution}

\end{solution}
\begin{exercise}[Opgave 5]
Kohærent lys med bølgelængden $\lambda$ sendes mod en skærm med to spalter med bredden $d /2$, i en indbyrdes afstand $d$ (målt fra center til center af de to spalter) som angivet på Figur 5. I afstanden $R$ fra skærmen med spalterne er anbragt en observationsskærm. $y$ angiver højden på observationsskærmen i forhold til midterlinjen ved $y=0$. Det antages at $R \gg y \gg d$ og at $\lambda < d /2}$.
\end{exercise}
\begin{subexercise}[a]
Opskriv et udtryk for intensiteten, $I\left( y \right) $, af det resulterende diffraktionsmønster på skærmen som funktion af $y$. Udtryk intensiteten ved $d$, $y$, $R$, $\lambda$, $I_0$, hvor $I_0= I\left( 0 \right) $ er intensiteten ved $y=0$.\\
Skitser intensiteten som funktion af $y$ (medtag de første 5 minima på hver side af $y = 0$. Vær specielt opmærksom på at angive korrekte positioner for $y$-værdier der resulterer i minimal intensitet.
\end{subexercise}
\begin{figure}[ht]
    \centering
    \incfig{opgave5}
    \caption{Opgave5}
    \label{fig:opgave5}
\end{figure}
\begin{solution}

\end{solution}
\end{document}
