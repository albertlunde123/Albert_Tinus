%%%%%%%%%%%%%%%%%%%%
%% SUPER PREAMBLE %%
%%%%%%%%%%%%%%%%%%%%

\documentclass[a4paper]{article}
\usepackage[utf8]{inputenc}
\usepackage[T1]{fontenc} % Fonts and stuff
\usepackage{amsmath, amsfonts, mathtools, amsthm, amssymb} % math

\usepackage{fancyhdr} % Header, Footer etc.
\usepackage{adforn}

\pagestyle{fancy}
\fancyhf{}
\fancyhead[R]{Albert Lunde, John Faxe Jensen}
\fancyfoot[C]{\adforn{17}\quad\thepage\quad\adforn{45}}

\renewcommand{\headrule}{%
	\hrulefill
	 {\quad\adforn{21}\adforn{10}\adforn{49}\quad}%
	\hrulefill}
\renewcommand{\footrulewidth}{0pt}

%% Margin Control %%

\def\changemargin#1#2{\list{}{\rightmargin#2\leftmargin#1}\item[]}
\let\endchangemargin=\endlist


%%%%%%%%%%%%%%%%%%%%%%%%%%%%%%%%%%%%%%%%%%%%%%%%%%%%%%%%%%

% figure support

\usepackage{import}
\usepackage{pdfpages}
\usepackage{transparent}
\usepackage{xcolor}

\newcommand{\incfig}[2][1]{%
    \def\svgwidth{#1\columnwidth}
    \import{figures/}{#2.pdf_tex}
}

\pdfsuppresswarningpagegroup=1

%%%%%%%%%%%%%%%%%%%%%%%%%%%%%%%%%%%%%%%%%%%%%%%%%%%%%%%%%%

\usepackage{tikzsymbols} % Symbols
\usepackage{mdframed} % Boxes around theorem environments
\usepackage{thmtools}


% Exercise environment with surrounding box

\declaretheoremstyle[
    name=Exercise,
    mdframed={
  skipabove=0pt,
  skipbelow=20pt,
  innerleftmargin=10pt,
  innerrightmargin=10pt,
  innerbottommargin=7pt}
]{thmsty}
\declaretheorem[style=thmsty,numbered=no]{exercise}

% Solution environment, with coffee cup

\newenvironment{solution}
 {\renewcommand\qedsymbol{\tikzsymbolsuse{Coffeecup}}\begin{proof}[Solution]}
 {\end{proof}}

% Subexercise enviroment
%  \newenvironment{subexercise}[1]
%  {
% 	\begin{changemargin}{1.0cm}{1.0cm}
% 	\textbf{(#1)}\em
% 	}{
	% \end{changemargin}
	% } 

 \newenvironment{subexercise}[1]
 {\noindent
	 \textbf{(#1)} \quad \adforn{10} \quad \em
 }{}

% Mathematical typesetting stuff.

 \newcommand{\dd}{\mathrm{\textbf{d}}}



\title{\vspace{-1cm}\efbox[margin = 15pt]{August - 2019}\vspace{-1cm}}
\author{}
\date{}

\begin{document}
\pagecolor{color1}
\maketitle
\thispagestyle{fancy}
\begin{exercise}[Opgave 1]
Vi betragter det elektrostatiske arrangement vist på Figur 1. En kugleskal
med indre radius $a$ og ydre radius $b$ bærer en sfærisk symmetrisk ladnings-fordeling, hvis volumensladningstæthed er givet ved.
\[
\rho\left( r \right) = \begin{cases}
	0 \quad &\left( r<a \right) \\
	\alpha r^2 \quad &\left( a<r<b \right) \\
	0 \quad &\left( r > b \right) 
\end{cases}
.\]
hvor $r$ er afstanden fra origo er $\alpha $ er en positiv konstant. Den dielektriske permittivitet lig $\epsilon_0$ overalt.
\end{exercise}
\begin{figure}[ht]
    \centering
    \incfig[0.4]{opgave1}
    \label{fig:opgave1}
\end{figure}
\begin{subexercise}[a]
Bestem den totale ladning Q båret af kugleskallen. Bestem retningen og størrelsen af det elektriske felt i områderne $r<a$,  $a<r<b$ og $r>b$.
\end{subexercise}
\begin{solution}

\end{solution}
\begin{subexercise}[b]
Angiv det elektriske potential i området $r>b$.
Bestem det arbejde der kræves for at flytte en punktladning $q\left( q>0 \right) $
langsomt fra uendelig langt væk $\left( r = \infty \right) $ til den ydre skal $\left( r=b \right) $.
\end{subexercise}
\begin{solution}

\end{solution}
\begin{exercise}[Opgave 2]
Vi betragter det elektriske kredsløb vist på Figur 2. Kredsløbet består af
en ideel emf kilde $\mathcal{E} $, to spoler med selvinduktans $L$ og tre modstande med
modstand $R$, som indikeret på figuren. Efter at have været åben i lang tid
sluttes kontakten til tiden $t=0$. Vi ser bort fra fælles induktans effekter.
\end{exercise}
\begin{figure}[ht]
    \centering
    \incfig[0.5]{opgave2}
    \label{fig:opgave2}
\end{figure}
\begin{subexercise}[a]
Bestem værdien af strømmen $i$, som løber i kredsløbet,
 \begin{itemize}
	\item til tiden $t = 0$, lige efter kontakten sluttes
	\item til tiden $t\to \infty$
\end{itemize}
\end{subexercise}
\begin{solution}

\end{solution}
\begin{exercise}[Opgave 3]
Punktladninger med ladning $q>0$ bevæger sig med konstant hastighed $\vec{\mathbf{v}} $
langs symmetriaksen (z-aksen) af en uendelig lang metalcylinder med radius $R$, som vist på figuren. Ladningernes volumentæthed $n\left( r \right) $ er givet ved
\[
n\left( r \right) = 
\begin{cases}
	n_0\left( 1 - r /R \right) \quad & \left( r<R \right) \\
	0 \quad & \left( r > R \right) 
\end{cases}
,\] 
Hvor $n_0$ er en positiv konstant og $r$ er afstand til $z$-aksen. Den magnetiske permeabilitet er $\mu_0$ overalt.
\end{exercise}
\begin{figure}[ht]
    \centering
    \incfig{opgave3}
    \label{fig:opgave3}
\end{figure}
\begin{subexercise}[a]
Angiv strømtætheden $\vec{\mathbf{J}} $ og bestem den strøm $I$ som løber gennem cylinderen.\\
Bestem retningen og størrelsen af det magnetiske felt i områderne $r>R$ og  $r<R$.
\end{subexercise}
\begin{solution}
\end{solution}
\begin{exercise}[Opgave 4]
En stationært halvbue-formet sløjfe med radius a ligger i (xy)-planen, som vist på Figur 4, og er påsat et uniformt magnetisk felt $\vec{\mathbf{B}}_0=B_0\left(\vec{\mathbf{j}} + \vec{\mathbf{k}}  \right)/\sqrt{2} $
hvor $B_0$ er en positiv konstant. Der løber ingen strøm i sløjfen til tiden $t<0$.
Efter $t=0$ aftager det eksterne magnetiske felts størrelse exponentialt, ifølge
\[
B_0\left( t \right)  = B_0e^{-t /\tau} \quad \left( t\ge 0 \right) 
,\]
hvor $\tau$ er en positiv konstant. Sløjfens modstand er R og vi ser bort fra dens selvinduktans. Den magnetiske permeabilitet er lig $\mu_0$ overalt
\end{exercise}
\begin{figure}[ht]
    \centering
    \incfig[0.6]{opgave4}
    \label{fig:opgave4}
\end{figure}
\begin{subexercise}[a]
Bestem retningen og størrelsen af den inducerede strøm i sløjfen til tiden $t\ge 0$.
\end{subexercise}
\begin{solution}

\end{solution}
\begin{exercise}[Opgave 5]
E-feltets størrelse og retning i en elektromagnetisk planbølge der udbreder
sig i vacuum er givet ved $\vec{\mathbf{E}} \left( x,t \right) = \vec{\mathbf{j}} E_i \cos\left( kx - \omega t \right) $. Der indsættes nu et
dielektrika med brydningsindex $n=2$ i den halvdel af rummet der er givet
ved $x<0$. $\mu = \mu_0$ i hele rummet.
\end{exercise}
\begin{subexercise}[a]
En del af den indkommende bølge reflekteres ved overgangen mellem vac-
uum og dielektrikaet som angivet i formel 35.16. Bestem amplituderne
af det elektriske og magnetiske felt i den reflekterede bølge, $E_r$ og $B_r$,
udtrykt ved $E_i$ . Bestem intensiteten af den indkommende og den reflek-
terede bølge og vis at intensiteten af den totale elektromagnetiske bølge
for $x<o$ er $I_{tot} = \frac{1}{2}\epsilon_0c \frac{8}{9}E_i^2$ .
(b) Opskriv udtryk for det elektriske og magnetiske felt for den transmit
\end{subexercise}
\begin{solution}

\end{solution}
\end{document}
